%  _____  _____  ___________   ____  ______  ___              _______     __________
%    |      |               |   | \ /         |              |       \   |          
%    |      |               |   |  \          |              |        |  |          
%    |------|               |   |---\------   |              |_______/   |------    
%    |      |    ____       |   |    \        |       ___    |           |          
%  __|__  __|__    \_______/   _|_   _\____    \_______/   __|__         |__________

%   __________  ___      ___   _____    _____  _____   ____  ____________  ____________    _______     _____    _____   __________     _______     
%  |             |        |    |  \      |      |    __/          |             |        /       \     |  \      |    |              |        \   
%  |             |        |    |   \     |      | __/             |             |       |         |    |   \     |    |              |        |   
%  |------       |        |    |    \    |      |/\__             |             |       |         |    |    \    |    |------        |________/   
%  |             |        |    |     \   |      |   \__           |             |       |         |    |     \   |    |              |        \   
%  |              \______/   __|__    \__|    __|__  __\__       _|_      ______|_____   \_______/   __|__    \__|    |__________  __|__       \__




% ------- Fortegn -------
% Indstillinger:
% Vaerdiet = 0  ->  + ax 
% Vaerdiet = 1  ->  ax
% 
% Parametre: 
% {a}{x} 
%     
% Logik: 
% 1x            ->  x. 
% -1x           ->  -x
% 0x            ->  ""
\newcommand{\fortegn}[2]{                       % {a}{x}
    \pgfmathparse{abs(#1 - int(#1)) > 0}             % 1.5 - 1 > 0 ? 
    % \pgfmathparse{#1 - int(#1) < 0}             % -1.5 - (-1) < 0 ? 
    \ifnum\pgfmathresult=1                      % Integer vs float check
    % Decimaltal
        \pgfmathparse{ifthenelse(#1<0, "", ifthenelse(\thevaerdiet==1, "", "+"))} % a < 0 ?    Første tal ? 
        \pgfmathresult #1#2 
    \else
    % Heltal
        \ifnum#1=0                              % a = 0 
        % Kommentar så kompiler virker.
        \else
            \ifnum#1=1
                \ifnum\thevaerdiet=1
                    #2                          % x
                \else 
                    + #2                        % +x
                \fi
            \else
                \ifnum#1=-1
                    - #2                        % -x
                \else
                    \ifnum#1>0
                        \ifnum\thevaerdiet=1
                            #1#2                % +ax
                        \else 
                            + #1#2              % ax
                        \fi
                    \else
                        #1#2                    % -ax
                    \fi
                \fi
            \fi
        \fi
    \fi
}
%\symbolEllerEj{2}  <- Ikke et symbol
%\symbolEllerEj{a}  <- Symbol
% Skal have formen \ifcsname #1\endcsname ikke \ifcsname #1 \endcsname
\newcommand{\erSymbol}[3]{ \ifcsname #1\endcsname                                           #2 \else #3 \fi }
                                            % |a - int(a)| > 0 ? 
\newcommand{\erDecimal}[3] { \pgfmathparse{abs(#1 - int(#1)) > 0} \ifnum\pgfmathresult=1    #2 \else #3 \fi }
  


\newcommand{\fortegnTo}[2]{                       % {a}{x}
    \erSymbol{#1}{
        % Er symbol
        - #1#2
    }{
        \erDecimal{#1}{
            % Er decimal
            \pgfmathparse{ifthenelse(#1<0, "", ifthenelse(\thevaerdiet==1, "", "+"))} % a < 0 ?    Første tal ? 
            \pgfmathresult #1#2 
        }{
            % Er heltal
            \ifnum#1=0                              % a = 0 
            % Kommentar så kompiler virker.
            \else
                \ifnum#1=1
                    \ifnum\thevaerdiet=1
                        #2                          % x
                    \else 
                        + #2                        % +x
                    \fi
                \else
                    \ifnum#1=-1
                        - #2                        % -x
                    \else
                        \ifnum#1>0
                            \ifnum\thevaerdiet=1
                                #1#2                % +ax
                            \else 
                                + #1#2              % ax
                            \fi
                        \else
                            #1#2                    % -ax
                        \fi
                    \fi
                \fi
            \fi
        }
    }
}









% ------- differensLigning -------

% Indstillinger:
% 
% Parametre: 
% {a}{b}{y}{x} 
%     
% Logik: 
% y[n] = sum(ay[n - i]) + b0x[n] + sum(bx[n - i])
\newcommand{\differensLigning}[4]{ 
    \register{0}{0, 1}                              % Startværdier sættes til 0. 
    y[n] = 
    \foreach \a [count=\i from 1] in {#1} {
        \fortegn{\a}{#3[n - \i]}
        \ifnum\thevaerdiet=1\register{0}{0, 0}\fi   % Første værdi sat 
    }
    \foreach \b [count=\i from 0] in {#2} {
        \ifnum\i=0
            \fortegn{\b}{#4[n]}
        \else   
            \fortegn{\b}{#4[n - \i]}
        \fi
        \ifnum\thevaerdiet=1\register{0}{0, 0}\fi   % Første værdi sat 
    }
}

% -------- filterZ --------

% Indstillinger:
% 
% Parametre: 
% {b_0, ..., b_N}{a_0, ..., a_N}
%     
% Logik: 
% (1 + 2z^-1)/(1 + 2z^-1 + 3z^-2)

% Hvad jeg ønsker. Fortegn virker dog ikke med decimaler.
\newcommand{\filterZto}[2]{
    \frac{
        \register{2}{0, 1}                                  % Første værdi mangler
        \foreach \b [count=\i from 0] in {#1} {
            \ifnum\i=0 \b \else \fortegnTo{\b}{z^{-\i}}\fi    % b eller bz^-i ? 
            \ifnum\vaerdifem=1\register{2}{0, 0}\fi
            % \ifnum\vaerdifem=1\register{2}{0, 0}\fi       % Værdi er sat
        }
    }{
        \register{2}{0, 1}                                  % Første værdi mangler
        \foreach \a [count=\i from 0] in {#2} {
            \ifnum\i=0 \a \else \fortegnTo{\a}{z^{-\i}}\fi    % a eller az^-i ? 
            \ifnum\vaerdiseks=1\register{2}{0, 0}\fi       % Værdi er sat
        }
    }
}
\newcommand{\filterZ}[2]{
    \frac{\foreach \b [count=\i from 0] in {#1}{
        \ifnum\i=0
            \b
        \else
            \pgfmathparse{ifthenelse(\b<0, "", "+")}\pgfmathresult \b z^{-\i}
        \fi}
    }{\foreach \a [count=\i from 0] in {#2}{
        \ifnum\i=0
            \a
        \else
            \pgfmathparse{ifthenelse(\a<0, "", "+")}\pgfmathresult \a z^{-\i}
        \fi}
    }
}





\newcommand{\convolutionNum}[3] { % View, Grænser, indhold. 
    \udfold{0}{#2}                                  % Udfolder grænser til registre. 
    y[n]   
    \ifnum#1=1                                      % View 1
        = x[n]\star h[n]
    \else 
        \ifnum#1=0                                  % View 2
            = \sum_{n = \thevaerdiet}^{\thevaerdito}{#3}
        \else                                       % View 3
            = x[n]\star h[n] = \sum_{n = \thevaerdiet}^{\thevaerdito}{#3}
        \fi
    \fi
}
\newcommand{\convolutionSym}[3] { % View, Grænser, (x, h) 
    \udfold{2}{#2}                                  % Udfolder grænser til registre. 
    \udfold{3}{#3}                                  % Udfolder grænser til registre.
    \ifnum#1=1                                      % View 1
        \vaerdisyv[k]\star \vaerdiotte[k]
    \else 
        \ifnum#1=0                                  % View 2
            \sum_{k = \vaerdifem}^{\vaerdiseks}{\vaerdisyv[k] * \vaerdiotte[n-k]}
        \else                                       % View 3
            \vaerdisyv[k]\star \vaerdiotte[k] = \sum_{k = \vaerdifem}^{\vaerdiseks}{\vaerdisyv[k] * \vaerdiotte[n-k]}
        \fi
    \fi
}
\newcommand{\convolution}[2]{
    y[n] = #1[n]\star #2[n] = \sum_{k = -\infty}^{\infty}{#1[k]* #2[n-k]}   
}

\newcommand{\infsum}[2]{\sum_{#1 = -\infty}^{\infty}{#2}}


\newcommand{\cosTilEksponentiel}[1]{ \frac{e^{j#1} + e^{-j#1}}{2} }
\newcommand{\sinTilEksponentiel}[1]{ \frac{e^{j#1} - e^{-j#1}}{2j} }

