% Opgaver fra kapitel 7 og frem

\begin{Opgaver}
    \begin{kapitel}[The Discrete Fourier Transform]
        \begin{Opgave}[Opgave 7.1 - FT og system beskrivelse, beregning mod fft (Vigtig !)]
            Let $xc(t) = 5e^{-10t}sin(20\pi*t)u(t)$.
            \begin{UnderOpgave}[Determine the CTFT \text{$X_c(j2\pi F)$ of xc(t)}]
                Bruger eulers identitet 
                \eulersIdentitetSin
                \[\frac{5}{2j}e^{-10t}(e^{j20\pi*t} - e^{-j20\pi*t})u(t)\]
                Jeg ser et dæmpende udtryk med en amplitude og jeg ser 2 harmoniske udtryk. 
                \[\frac{5}{2j}e^{-10t}u(t) * (e^{j20\pi*t} - e^{-j20\pi*t})\]
                Gange i tids domænet er det samme som convolution i frekvens domænet, men med en $\frac{1}{2\pi}$ faktor på. \\
                Første udtryk: 
                \[\frac{5}{2j}e^{-10t}u(t) \transformation{F} \frac{5}{2j} \frac{1}{10 + j\omega}\]
                Andet og tredje udtryk: 
                \[e^{j20\pi*t} \transformation{F} 2\pi\delta(\omega - 20\pi)\]
                \[-e^{-j20\pi*t} \transformation{F} -2\pi\delta(\omega + 20\pi)\]
                Da det bare er to impulser i frekvens spektret er de forholdsvis simple at forestille sig convolution med. 
                Omkring hvert af disse punkter vil første udtryk have origo i. Så den bliver forskudt fra w = 0 til $\omega \pm = 20$
                \[X_c(j2\pi F) = \frac{1}{2\pi} * \int_{-\infty}^{\infty}{X(j\theta)H(j(\omega - \theta)) d\theta}\]
                \[X_c(j\omega) = \frac{5}{2\pi * 2j} (\frac{1}{10 + j(\omega - 20\pi)} - \frac{1}{10 + j(\omega + 20\pi)})\]
                Fundet vha egenskaben at: 
                \[x(t)\star\delta(t - t_0) \rightarrow x(t - t_0)\]
                Impulser forsinker eller fremskyder bare signalet.
                Det burde kunne beskrive det så præcist analytisk som muligt
                \vspace{150pt}
            \end{UnderOpgave}
            \begin{UnderOpgave}[Plot magnitude and phase of \text{$X_c(j2\pi F)$ over \text{$—50 < F < 50$} Hz}]
                \figtoogtres{0.25}
            \end{UnderOpgave}

            \begin{UnderOpgave}[Use the fft function to approximate the CTFT computation]
                Choose a sampling rate to minimize aliasing and the number of samples to capture the signal wave-
                form. Plot magnitude and phase of your approximation and compare it with the
                plot in (a) above.\\
                Jeg ser at filtret stort set er 0 for $\omega = \pm 314, F = \pm 50$
                Så skal jeg bare opfylde nyquist raten med den. 
                $F_s > 2F_h = 2 * 50Hz$
                Så en sampling frekvens på 100Hz virker ikke dårlig. $F_s = 100Hz$
                $T_s = 1/F_s = 1/100$ så det bliver mine steps.
                Sinus delen af funktionen opnår 10 perioder indenfor 1 sekund. Et halv sekund skulle da være rigeligt for at få sig et billede. Måske skal jeg endda tættere på pga dæmperen. 
                \figtreogtres{0.30}\\\\
                Der ses at magnituden stort set er den samme. Jeg har lidt en ting med at misse amplituderne, det er ikke første gang. 
                Min mangler jo i hvert fald en general forstærkning, men formen er meget rigtig.\\              
                \color{red}Gammelt plot: Fasen for fft er lidt hulter til bulter, men det er måske på grund af usikkerheder. 
                Hvis man kørte frekvensen gennem et lavpass filter, så tror jeg, at fasen også nogenlunde vil vise det samme billede. 
                \color{black}Vigtig ændring jeg gjorde mig. Jeg fandt ud af, at det var antikausulitet som gjorde det.
                Jeg havde lavet mit signal fra -0.5, 0.5 sekunder. Det er jo egentlig ikke det jeg ønsker det meste af tiden hellere. 
                Nu ligner fasen mere hvordan den beregnede fase er i t = 0, ... 1 sekund
            \end{UnderOpgave}
        \end{Opgave}
        \begin{Opgave}[Opgave 7.3 - DTFT mod FFT]
            Let $x[n] = n(0.9)^nu[n]$
            \begin{UnderOpgave}[Determine the DTFT \text{$\tilde{X}(e^{j\omega})$} of \text{$x[n]$}]
                Det virker som om at tilde notationen bare er en måde at udtrykke, at spektret er periodisk. 
                Jeg har udledt den på 3 forskellige måder, som kan findes under udledninger. 
                Den kunne blive løst for på flere måder. Den geometriske serie og differentations egenskaben er hvad jeg har taget brug af. 
                \[x[n] \transformation{F} X(e^{j\omega}) = \frac{ae^{-j\omega}}{(1 - ae^{-j\omega})^2}\]
                Og så længe a ikke er 1, så er det en løsning.
                \[x[n] \transformation{F} X(e^{j\omega}) = \frac{0.9*e^{-j\omega}}{(1 - 0.9*e^{-j\omega})^2}\]
            \end{UnderOpgave}

            \begin{UnderOpgave}[Choose first N = 20 samples of \text{$x[n]$}]
                and compute the approximate DTFT $\tilde{X}_N(e^{j\omega})$ using the fft function. 
                Plot magnitudes of $\tilde{X}(e^{j\omega})$ and $\tilde{X}_N(e^{j\omega})$ in one plot and
                compare your results.\\
                \figfireogtres{0.3}
                Min beregnede er lavet ud fra en formel. Måske gør det, at den ikke konvergere så hurtigt som den fft'erede. \\
                Fasen er så ca 180 graders forskudte. 

            \end{UnderOpgave}
            
            \begin{UnderOpgave}[Repeat part (b) using N = 50.]
                \figfemogtres{0.3}\vspace{60pt}
            \end{UnderOpgave}

            \begin{UnderOpgave}[Repeat part (b) using N = 100.]
                \figseksogtres{0.3}
            \end{UnderOpgave}      
            Jeg ser at min beregnede funktions magnitude konvergere mod den fft'erede. \\
            Den fft'eredes fase konvergere så mod min beregnedes fase.
            \color{teal} Jeg havde lidt problemer da jeg brugte frekvenserne fra np.fft.fftfreq. Da jeg selv lavede w fik jeg resultatet som jeg ønskede. 
            \color{black}
        \end{Opgave}
        \begin{Opgave}[Opgave 7.4 - Matrix beregning (Kommet hertil, har ikke kunnet bevise bevis)]
            Let $W_N$ be the N x N DFT matrix.
            \begin{UnderOpgave}[Determine \text{$W_N^2$}, and verify that it is equal to \text{$NJ_N$} where \text{$J_N$} is known as a flip
                matrix. Describe this matrix and its effect on \text{$J_x$} product.]
                Jeg har spurgt chatten og søgt lidt på nettet for at eftertjekke, og det virker som om at matricen 
                \[ J_N = 
                    \begin{vmatrix}
                        0     & \vdots & 0     & 0     & 1 \\
                        0     & \vdots & 0     & 1     & 0 \\
                        0     & \vdots & 1     & 0     & 0 \\
                        \dots & \dots  & \dots & \dots & \dots \\
                        1     & \vdots & 0     & 0     & 0
                    \end{vmatrix}
                \]
                Eller bare enhver anden matrice af NxN hvor antidiagonalen har 1'ere. \\
                Afhængigt af om man sætter den højre eller venstre i en matrix multiplikation, så vil den medføre et flip horizontalt eller vertikalt. \\
                Chatten siger, at $N J_n$ skulle give en horizontal flip i den rækkefølge. 
                en N er bare en skalar, så i det her tilfælde vil den bare være matricen skaleret. 
                For $W_N^2$ så bliver matricerne sat op til matrix multiplikation, og da vil dens kolonner blive ganget med rækkerne for at danne den nye matrice.
                \[ W_N = 
                    \begin{vmatrix}
                        w_{11} & w_{12} & w_{13} \\
                        w_{21} & w_{22} & w_{23} \\
                        w_{31} & w_{32} & w_{33}  
                    \end{vmatrix}
                \]
                \[ W_N^2  = 
                    \begin{vmatrix}
                        w_{11} & w_{12} & w_{13} \\
                        w_{21} & w_{22} & w_{23} \\
                        w_{31} & w_{32} & w_{33}  
                    \end{vmatrix}
                    @ 
                    \begin{vmatrix}
                        w_{11} & w_{12} & w_{13} \\
                        w_{21} & w_{22} & w_{23} \\
                        w_{31} & w_{32} & w_{33}  
                    \end{vmatrix}
                    = 
                    \begin{vmatrix}
                        w_{11}w_{11} + w_{12}w_{21} +w_{13}w_{31} & w_{11}w_{12} + w_{12}w_{22} +w_{13}w_{32} & w_{11}w_{13} + w_{13}w_{23} +w_{13}w_{33} \\
                        w_{21}w_{11} + w_{22}w_{21} +w_{23}w_{31} & w_{21}w_{12} + w_{22}w_{22} +w_{23}w_{32} & w_{21}w_{13} + w_{23}w_{23} +w_{23}w_{33} \\
                        w_{31}w_{11} + w_{32}w_{21} +w_{33}w_{31} & w_{31}w_{12} + w_{32}w_{22} +w_{33}w_{32} & w_{31}w_{13} + w_{33}w_{23} +w_{33}w_{33} \\ 
                    \end{vmatrix}
                \]
                Vindue værdierne er ikke andet en kompleks eksponentielle med forskellige frekvenser 
                \[w_{k = 2, l = 2} = (e^{-j\omega_0})^{2*2}\]
                Så lad mig se på hvad det vil give, en kolonne af gangen.\\\\
                Kolonne 1 
                \[\begin{vmatrix}
                    w_{11}w_{11} + w_{12}w_{21} +w_{13}w_{31} \\
                    w_{21}w_{11} + w_{22}w_{21} +w_{23}w_{31} \\
                    w_{31}w_{11} + w_{32}w_{21} +w_{33}w_{31} \\
                \end{vmatrix}
                = 
                \begin{vmatrix}
                    e^{-0*0 * j\omega}e^{-0*0 * j\omega} + e^{-0*1 * j\omega}e^{-1*0 * j\omega} + e^{-0*2 * j\omega}e^{-2*0 * j\omega} \\
                    e^{-1*0 * j\omega}e^{-0*0 * j\omega} + e^{-1*1 * j\omega}e^{-1*0 * j\omega} + e^{-1*2 * j\omega}e^{-2*0 * j\omega} \\
                    e^{-2*0 * j\omega}e^{-0*0 * j\omega} + e^{-2*1 * j\omega}e^{-1*0 * j\omega} + e^{-2*2 * j\omega}e^{-2*0 * j\omega} \\
                \end{vmatrix}\]
                Så det er en masse 1'ere, men det vidste jeg godt inden. Lad mig se på kolonner som ikke er 1'ere.
                \[\begin{vmatrix}
                    w_{11}w_{11} + w_{12}w_{21} +w_{13}w_{31} \\
                    w_{21}w_{11} + w_{22}w_{21} +w_{23}w_{31} \\
                    w_{31}w_{11} + w_{32}w_{21} +w_{33}w_{31} \\
                \end{vmatrix}
                = 
                \begin{vmatrix}
                    1 + 1 + 1 \\
                    1 + 1 + 1 \\
                    1 + 1 + 1
                \end{vmatrix}\]\\\\


                Kolonne 2
                \[\begin{vmatrix}
                    w_{11}w_{12} + w_{12}w_{22} +w_{13}w_{32} \\
                    w_{21}w_{12} + w_{22}w_{22} +w_{23}w_{32} \\
                    w_{31}w_{12} + w_{32}w_{22} +w_{33}w_{32} \\
                \end{vmatrix}
                = 
                \begin{vmatrix}
                    e^{-0*0 * j\omega}e^{-0*1 * j\omega} + e^{-0*1 * j\omega}e^{-1*1 * j\omega} + e^{-0*2 * j\omega}e^{-2*1 * j\omega}  \\
                    e^{-1*0 * j\omega}e^{-0*1 * j\omega} + e^{-1*1 * j\omega}e^{-1*1 * j\omega} + e^{-1*2 * j\omega}e^{-2*1 * j\omega}  \\
                    e^{-2*0 * j\omega}e^{-0*1 * j\omega} + e^{-2*1 * j\omega}e^{-1*1 * j\omega} + e^{-2*2 * j\omega}e^{-2*1 * j\omega}  \\
                \end{vmatrix}\]
                \[\begin{vmatrix}
                    w_{11}w_{12} + w_{12}w_{22} +w_{13}w_{32} \\
                    w_{21}w_{12} + w_{22}w_{22} +w_{23}w_{32} \\
                    w_{31}w_{12} + w_{32}w_{22} +w_{33}w_{32} \\
                \end{vmatrix}
                = 
                \begin{vmatrix}
                    1 + 1 + 1                                 \\
                    1 + e^{-2 * j\omega} + e^{-4 * j\omega}   \\
                    1 + e^{-3 * j\omega} + e^{-6 * j\omega}   
                \end{vmatrix}\]\\\\


                Kolonne 3 
                \[\begin{vmatrix}
                    w_{11}w_{13} + w_{13}w_{23} +w_{13}w_{33} \\
                    w_{21}w_{13} + w_{23}w_{23} +w_{23}w_{33} \\
                    w_{31}w_{13} + w_{33}w_{23} +w_{33}w_{33} \\ 
                \end{vmatrix}
                = 
                \begin{vmatrix}
                    1 + 1 + 1                               \\
                    1 + e^{-4 * j\omega} + e^{-6 * j\omega} \\
                    1 + e^{-6 * j\omega} + e^{-8 * j\omega} \\
                \end{vmatrix}\]
                

                \[W_N^2 = \begin{vmatrix}
                    3 &                                       3 &                                       3 \\                 
                    3 & 1 + e^{-2 * j\omega} + e^{-4 * j\omega} & 1 + e^{-4 * j\omega} + e^{-6 * j\omega} \\
                    3 & 1 + e^{-3 * j\omega} + e^{-6 * j\omega} & 1 + e^{-6 * j\omega} + e^{-8 * j\omega} \\
                \end{vmatrix}\]
                For at jeg skulle have bevist det, så skulle det her have været lige med N på antidiagonalen og 0 ellers.
                Det er ikke det jeg ser, så derfor kan jeg ikke bevise det. 
                Jeg kan stadigvæk ikke se hvordan det her bevis skulle kunne bevises. 

            \end{UnderOpgave}
            \begin{UnderOpgave}[Show that \text{$W_N^4, = N^2I_N$}. Explain the implication of this result.]
                Jeg har ikke kunnet udlede sidste bevis. Men antaget at det holder så er
                \[W_N^4 = (W_N^2)@(W_N^2) = (N*J_N) @ (N*J_N) = \sum_{k = 0}^{N} \sum_{i = 0}^{N} N * N, k = i \]
                Ved matrix multiplikation, hvis den første matrice har en af dens nedereste elementer, så bliver der ganget med den samme, spejlvendt om diagonalen. 
                N * N, k, i = (3, 4), (4, 3). \\
                Jeg ser dog stadigvæk, at den kommer til at være på antidiagonalen. 
                Jeg kunne også have vist det ved den associative egenskab ved multiplikation af skalarer på matricer. 
                \[c(dA) = (cd)A\]
                \[W_N^4 = (W_N^2)@(W_N^2) = N^2(J_N @ J_N)\]
                \[N^2(J_N @ J_N)_{i, j}\] 
                \[N^2(J_N @ J_N)_{0, N} = N^2*(J_N{N, 0} * J_N{0, N})\]
                \[N^2(J_N @ J_N)_{1, N - 1} = N^2*(J_N{N - 1, 1} * J_N{1, N - 1})\]
                \[...\]
                Igen jeg ser det stadigvæk som en antidiagonal matrice. 

            \end{UnderOpgave}
            \begin{UnderOpgave}[Using MATLAB determine eigenvalues of \text{$\frac{W_N}{\sqrt{N}}$} for \text{$4 \geq N \geq 10$}. Some of the
                eigenvalues may be repeated. Can you guess a general rule for the multiplicity of
                eigenvalues as a function of N?]
                Jeg har lavet et script til det i min beregningsfil 
                
            \end{UnderOpgave}
        \end{Opgave}
        \begin{Opgave}[Opgave 7.5 Determine the N point DFTs of the following sequences defined over \text{$0\geq n\geq N$}]
            Jeg har fået nogle funktioner, og så skal jeg lave DFT på dem. Jeg opstillerne funktionerne i et bibliotek. \\
            funktioner = { \\
                "a" : lambda n : 4 - n, \\
                "b" : lambda n : 4 * np.sin(0.2 * np.pi * n), \\
                "c" : lambda n : 6 * (np.cos(0.2 * np.pi * n)**2), \\
                "d" : lambda n : 5 * ((0.8)**n), \\
                "e" : lambda n : np.array([3 if i mod 2 == 0 else -2 for i in range(len(n))]) \\
            } 
            N = [8, 10, 10, 16, 20]
            \[X = Wx\]
            Så jeg laver matrix multiplikation i den rækkefølge
            \figsyvogtres{0.3}\\
            Generelt er resolutionen meget dårlig, så derfor er spektret meget kantet.\\
            Noget andet jeg vil sige om plottet er, at funktionerne er beregnet ud fra den N værdi de fik tildelt. Derfor har jeg zero paddet resten af den spektrum, så det passede in i en general w
            Hvis det skulle være helt rigtigt, så fylder hvert plot jo hele spektret, men forskellige oplysninger, så det skal man forestille sig at de gør.\\
            Svararket har også løst for den analytisk, det kunne man også, men nu valgte jeg at løse for den numerisk. 
        \end{Opgave}
        \begin{Opgave}[Opgave 7.6 - Show that the DFT coefficients...]
            $X[k]$ are the projections of the signal $x[n]$ on the DFT (basis) vectors $w_k$. 
            \[X = Wx\]
            \[N\times N \tab{0} \times \tab{0} N\times 1 \tab{0} \rightarrow \tab{0} N\times 1\]
            Hver række i W matrixen bliver ganget på kolonnen af inputs.
            \[X_i = W_{i} * x\]
            \[X_i = \sum_{j = 0}^{N}W_{i, j} * x_j\]
            Efter hans notation: 
            \[X_k = W_{k} * x = \sum_{j = 0}^{N}W_{k, j} * x_j\]
            Han kalder det godt nok projektion, men jeg vil nu kalde det for et prik produkt, men jeg er godt klar over, at der er nogen der også kalder det for en projektion. 
            Jeg blev lidt forvirret af hans notation, for han beskriver det nærmest som om, at $X[k]$ er en vektor. Jeg vil nu mene at det er en skalar, og det samme vil chatten.            
            
        \end{Opgave}
        \begin{Opgave}[Opgave 7.24 - Udledning af velkendte vinduer ud fra den generelle vindue formel]
            The CTFT $W(j\Omega)$ of a generic window function $w(t)$is given in (7.189) in which a
            and b are design parameters and $W_R(j\Omega)$ is the CTFT of the rectangular window.
            \begin{UnderOpgave}[For the choice of \text{$a = b = 0.5$} and using ICTFT show that the resulting Hann
                window function is given by eq. 7.190.]
                Formel 7.189: 
                \[W_c(j\omega)=a W_{\mathrm{R}}(j\Omega)+b W_{\mathrm{R}}(j(\Omega-2\pi/T_{0}))+b W_{\mathrm{R}}(j(\Omega+2\pi/T_{0}))\]
                \[W_c(j\omega)=\frac{1}{2} * (W_{\mathrm{R}}(j\Omega)+ W_{\mathrm{R}}(j(\Omega-2\pi/T_{0})) + W_{\mathrm{R}}(j(\Omega+2\pi/T_{0})))\]
                Det lugter lidt af konvolution i frekvens spektret. Det i tids domæne er multiplikation
                \[x(t)h(t) \Leftrightarrow \frac{1}{2\pi}\int_{-\infty}^{\infty}{X(j\theta) * H(j*(\omega - \theta)) d\theta}\]
                Her mangler jeg dog en delt med pi faktor. 
                Men hvis x(t) i det her tilfælde er vindue funktionen, så må h(t) være en noget der i frekvens spektret danner 3 impulser. 
                To impulser symmetrisk om $\omega = 0$ er en cos funktion i tids domænet, så der har jeg en af dem fundet ud af. Den sidste er så bare en impuls, og den er en dc forstærkning i tids domænet. 
                \[cos(\omega t) \transformation{F} = \pi*(\delta(\omega - \omega_0) + \delta(\omega + \omega_0))\]
                \[x(t) = 1 \transformation{F} = 2\pi\delta(\omega)\]
                For at de alle sammen skal være skaleret ligeligt, så har jeg at 
                \[a * x(t) = 1 \transformation{F} = a * 2\pi\delta(\omega) = \pi\delta(\omega)\]
                Og derfor bliver dc værdien nødt til at skaleres med en halv.
                Så nu har jeg et udtryk for begge dele af produktet i tid.
                \[====================\]
                \[w_han(t) = (cos(\omega_0 t) + 1/2) * (w_R(t))\]
                \[====================\]
                Giver det mening? 
                \[\frac{1}{2\pi} * \pi = \frac{1}{2}\]
                Som var den koefficient jeg fik for vinduefunktionen, så den burde være tilstrækkeligt med en koefficent foran på 1.
                I bogen så er en hanning filter 0.5*cos\\
                Der er fejl i opgaven. De løser den selv med en hel cos, så det er rigtigt nok
            \end{UnderOpgave}
            
            \begin{UnderOpgave}[For the choice of \text{$a = 0.54, b = 0.23$} and using ICTFT show that the resulting
                Hamming window function is given by eq. 7.191]
                Igen formel 7.189: 
                \[W_c(j\omega)=a W_{\mathrm{R}}(j\Omega)+b W_{\mathrm{R}}(j(\Omega-2\pi/T_{0}))+b W_{\mathrm{R}}(j(\Omega+2\pi/T_{0}))\]
                I konvolution: 
                \[(\frac{1}{2\pi} * (k_1 * \pi[\delta(\omega - \omega_0) + \delta(\omega + \omega_0)] + k_2 * 2\pi\delta(\omega))\star W_R(j\omega))\]
                Og jeg har at 
                \[k_2 * 2\pi * \frac{1}{2\pi} = 0.54, \tab{0} k_2 = 0.54\]
                \[k_1 * \pi * \frac{1}{2\pi} = 0.23, \tab{0} k_1 = 0.46\]
                Lad mig eftertjekke om det giver mig standard formlen tilbage. 
                \[(\frac{1}{2\pi} * (0.46 * \pi[\delta(\omega - \omega_0) + \delta(\omega + \omega_0)] + 0.54 * 2\pi\delta(\omega))\star W_R(j\omega))\]
                \[(0.23 * [\delta(\omega - \omega_0) + \delta(\omega + \omega_0)] + 0.54 * \delta(\omega))\star W_R(j\omega)\]
                Og det er jo netop standard formen med substitution af a og b værdier.
                \[x(t)h(t) = (k_1 * cos(\omega_0 t) + k_2) * w_r(t) \]
                \[x(t)h(t) = (0.46 * cos(\omega_0 t) + 0.54) * w_r(t) \]
                \[=======================\]
                \[x(t)h(t) = (0.54 + 0.46 * cos(\omega_0 t)) * w_r(t) \]
                \[=======================\]
                Nu er den faktisk helt identisk til formel 7.191, det eneste jeg ekstra jeg kunne, var at substituere for $\omega_0$, men det behøves jeg ikke. 


            \end{UnderOpgave}
        \end{Opgave}
    \end{kapitel}
    \begin{kapitel}[Computation of the Discrete Fourier Transform]
        \begin{Opgave}[Twiddle factors]
            \begin{UnderOpgave}[Confirm equations (8.5) and (8.6).]
                Formel 8.5 forklare om periodiciteten. Den forklare at vinduet er periodisk i k og n. 
                \[W_{N}^{kn} = W_{N}^{k(n+N)}=W_{N}^{(k+N)n}\]
                Til den kan jeg gå tilbage til definitionen på $W_N$ 
                \[W_N^{kn} = e^{-j\omega_0 * (kn)}\]
                Men for den mere generaliserende måde, så glemmer vi vinkelfrekvensen og ser på opløsningen vi har i vores data.\\
                For N datapunkter, så deler vi spektret ud på N punkter. Det sikre normalisering. Da er: 
                \[W_N : e^{-j * (\frac{2\pi}{N}) * (kn)}\]
                Det var sådan, at vi definerede den normaliserede vindue matrix. 
                \[W_N^{k*(n + N)} = e^{-j * (\frac{2\pi}{N}) * (k*(n + N))}\]
                \[W_N^{k*(n + N)} = e^{-j * (\frac{2\pi}{N}) * kn  -j * (\frac{2\pi}{N}) * kN}\]
                \[W_N^{k*(n + N)} = e^{-j * (\frac{2\pi}{N}) * kn}e^{-j * (\frac{2\pi}{N}) * kN}\]
                \[W_N^{k*(n + N)} = e^{-j * (\frac{2\pi}{N}) * kn}e^{-j * 2\pi * k}\]
                \[e^{-j * 2\pi * k} = 1\]
                For alle k'ere. 
                \[W_N^{kn} = e^{-j * (\frac{2\pi}{N}) * kn}\]
                Og derfor er den periodisk i N.
                Hvad med for n faktoreret? 
                \[W_N^{n*(k + N)} = e^{-j * (\frac{2\pi}{N}) * (n*(k + N))}\]
                \[W_N^{n*(k + N)} = e^{-j * (\frac{2\pi}{N}) * kn}e^{-j * (\frac{2\pi}{N})nN}\]
                \[W_N^{n*(k + N)} = e^{-j * (\frac{2\pi}{N}) * kn}e^{-j * 2\pi * n}\]
                \[e^{-j * 2\pi * n} = 1\]
                For alle n'ere.
                Derfor er k og n begge periodiske i N
                \vspace{40pt}\\
                Formel 8.6 forklare om komplex konjugeret symmetri. Den inddrager også lige en del af periodiciteten. 
                \[W_N^{k*(N - n)} = W_N^{-kn} = (W_N^{kn})^\star\]
                Vindue funktionen er ikke andet end en kompleks exponentiel funktion. Den konjugerede vil være det samme som fortegnet i eksponenten vendt. 
                \[W_N^{-kn} = e^{-j * (\frac{2\pi}{N}) * (-kn)} = e^{j * (\frac{2\pi}{N}) * (kn)}\]
                For at bevise det, tager jeg det også på trigonometrisk repræsentation.
                \[W_N^{-kn} = cos((\frac{2\pi}{N}) * (kn)) + jsin((\frac{2\pi}{N}) * (kn))\]\\
                Og lad mig så repræsentere det normale vindue trigonometrisk også 
                \[W_N^{kn} = e^{-j * (\frac{2\pi}{N}) * (kn)}\]
                \[W_N^{kn} = e^{j * (\frac{2\pi}{N}) * (-kn)}\]
                \[W_N^{kn} = cos(-(\frac{2\pi}{N}) * kn) + jsin(-(\frac{2\pi}{N}) * kn) \]
                \[cos(x) = cos(-x), sin(-x) = -sin(x)\]
                \[W_N^{kn} = cos(\frac{2\pi}{N} * kn) - jsin(\frac{2\pi}{N} * kn)\]
                Som vitterligt bare har et modsat fortegn på den komplekse del, og det er jo en kompleks konjugering. Derfor gælder der at: 
                \[W_N^{k*(N - n)} = W_N^{-kn} = (W_N^{kn})^\star\]
            \end{UnderOpgave}

            \begin{UnderOpgave}[Develop an expression for \text{$W_N^{(N-k)n}$} similar to equations (8.5) and (8.6).]
                Her er det egentlig bare fordi at vinduet er konjugeret, og jeg skal bevise, at periodiciteten stadigvæk gælder. 
                Hvis jeg gør det, så har jeg jo lige ovenover bevist konjugeringen, så med periodiciteten vil det være det samme. 
                \[W_N^{(N - k)n} = e^{-j * (\frac{2\pi}{N}) * ((N - k)n)}\]
                \[W_N^{(N - k)n} = e^{-j * (\frac{2\pi}{N}) * (-kn)}e^{-j * (\frac{2\pi}{N}) * (Nn)}\]
                \[W_N^{(N - k)n} = e^{-j * (\frac{2\pi}{N}) * (-kn)}e^{-j 2\pi n}\]
                \[e^{-j 2\pi n} = 1\] 
                for alle n'er. Og derfor gælder periodiciteten også ved kompleks konjugering. 
                På den måde har jeg så bevist, at peridiciteten også holder i den ligning som jeg sluttede sidste opgave af med. 
            \end{UnderOpgave}

            \begin{UnderOpgave}[Compute \text{$W_N^{N/2}$}]
                \[W_N^{N/2} = e^{-j * (\frac{2\pi}{N}) * (N/2)}\]
                \[===========\]
                \[W_N^{N/2} = e^{-j * \pi} = -1\]
                \[===========\]
            \end{UnderOpgave}

            \begin{UnderOpgave}[How are \text{$W_N^2$}, and \text{$W_{N/2}$} related?]
                De er relaterede ved, at normaliseringen er ud fra N. Hvis N bliver halvt så stor, så vil det være det samme som at gange med to.
                \[W_N^{2} = e^{-j * (\frac{2\pi}{N}) * 2}\]
                \[W_{N/2} = e^{-j * (\frac{2\pi}{\frac{N}{2}})} \]
                \[W_{N/2} = e^{-j * (\frac{2\pi}{N}) * 2 }\]
                \[W_N^{2} = W_{N/2}\]
                
            \end{UnderOpgave}
            \begin{UnderOpgave}[Is there a simple relation between \text{$W_{16}^9$} and \text{$W_{16}$}? Can this relation be generalized?]
                Lad mig se det an. 
                \[W_{16} = e^{-j * (\frac{2\pi}{16})}\]
                \[W_{16}^9 = e^{-j * (\frac{2\pi}{16}) * (9)}\]
                \[W_{16}^9 = e^{-j * 2\pi * \frac{9}{16}}\]
                Så er hypotesen, at man kan generalisere det sådan her: 
                \[W_i^j = e^{-j * 2\pi * \frac{j}{i}}\]
                Gælder det også for
                \[W_{14}^4 = e^{-j * (\frac{2\pi}{14})} * 4\]
                \[W_{14}^4 = e^{-j * 2\pi (\frac{4}{14})}\]
                Ja det gør det. Så generaliseringen er, at: 
                \[W_i^j = e^{-j * 2\pi * \frac{j}{i}}\]
                Og jo tættere de er på hinanden, jo tættere er de på en konstant værdi. 
                Så husker jeg fra det diskrete spektrum, at så gentager de sig derefter, så. 
                \[W_i^j = W_i^{j + i} = e^{-j * 2\pi * \frac{j}{i}}\]
                For mit tilfælde: 
                \[W_{14}^4 = W_{14}^{18} = e^{-j * 2\pi (\frac{4}{14})} = e^{-j * 2\pi (\frac{18}{14})}\]
            \end{UnderOpgave}
        \end{Opgave}
        \begin{Opgave}[Dual-tone multifrequency (DTMF) analysis]
            Opgaven handler om mapping af en sum af lav og høj frekvenser som laver en keypad grid. \\
            \figotteogtres{0.3}

            Være opmærksom på skrive fejlene. Der bliver beskrevet 7 frekvenser til at repræsentere 12 karakterer. 
            I virkeligheden, så er der 16 karakterer og det kræver 8 frekvenser.
            \begin{UnderOpgave}[Write code to generate samples of high and low frequencies of one-half
    second duration given a symbol “S”]
                Sådan som jeg forstår det, så skal jeg bruge det grid som jeg har fået som sketch, og vælge mine egne karakterer. 
                Så lad mig starte fra venstre af fra toppen. Så dens 1'er bliver min S'er. 
                \[S \rightarrow (697, 1209)Hz\]

            \end{UnderOpgave}
            \begin{UnderOpgave}[Make example sequences with random symbols and generate audible out-
    put. Comment on the sounds you can hear.]
                Jeg har omstruktureret keyboardet, så det står for de første 16 bogstaver i alfabetet. 
                Så har jeg valgt at spille frekvensen "Kongen". 
                Lyden er meget monotomt, tror jeg? Er det ikke hvor man taler meget i sådan samme toneleje? 
                Og sådan er det jo fordi den kun skifter tone efter hvert halve sekund. 

            
            \end{UnderOpgave}
            \begin{UnderOpgave}[Make FFT's of your examples sequences and plot the spectra. Make sure
    your spectra has the right frequency axis. Comment on the spectrum if
    you use the full example sequence or an small segment of the sequence.]
                \figniogtres{0.4}
            Hvad jeg fandt ud af her er, at det er en dum idé at prøve at analysere hele lyden på en gang. Der er jo uendelig store skift imellem dem ved transition, så derfor prøver den at lave frekvens komponenter ud fra dem. 
            Ellers så fandt jeg lidt ud af den bedste måde at samle tonerne, splitte dem ind til pakker med symboler, og at 
            numpy laver fft gennem kolonnerne, så (N, 1) vektorer er no go, den fatter det ikke. Lav (1, N) transformationer. Frekvenserne ser ud til at være ca. dem de blev lavet ud fra. \\
            konfiguration = {\\
                "A" : (697, 1209), \\
                "B" : (697, 1336), \\
                "C" : (697, 1477), \\
                "D" : (697, 1633), \\
                "E" : (770, 1209), \\
                "F" : (770, 1336), \\
                "G" : (770, 1477), \\
                "H" : (770, 1633), \\
                "I" : (852, 1209), \\
                "J" : (852, 1336), \\
                "K" : (852, 1477), \\
                "L" : (852, 1633), \\
                "M" : (941, 1209), \\
                "N" : (941, 1336), \\
                "O" : (941, 1477), \\
                "P" : (941, 1633)  \\    
            }
            \end{UnderOpgave}
            

            \begin{UnderOpgave}[Implement the Goertzel algorithm from chapter 8.7.1 and make plots that
    document the implementation works as intended.]
                \[Y_k(z)/X_k(z) = H_k(z) = \frac{1}{1 - 2cos(2\pi k / N)z^{-1} + z^{-2}}[1 - W_N^k z^{-1}]\]
                \[Y_k(z) = X_k(z) \frac{1}{1 - 2cos(2\pi k / N)z^{-1} + z^{-2}}[1 - W_N^k z^{-1}]\]


            \end{UnderOpgave}
            \begin{UnderOpgave}[On blackboard, you can find a file “DTMF.dat”. Load this file and decode
    the message within.]
                Der var ingen fil, men online fandt jeg en fil som test. Jeg har testen den samt filen jeg selv lavede. Det hele er i python dokumentet.

            \end{UnderOpgave}
        \end{Opgave}
    \end{kapitel}
    \begin{kapitel}[Structures for discrete-time systems]

        \begin{Opgave}[Opgave 9.1]
            Et DT system er beskrevet ud fra signal flow diagrammet
            \fighalvfjerds{0.3}
            \begin{UnderOpgave}[Determine the difference equation relating output yn to the input xn]
                Hvis jeg går med knudepunktet under buen, og kalder den for $S[n]$ så har jeg at: 
                \[S[n] = x[n] + \frac{1}{3}S[n - 1]\]
                Så har jeg dens knudepunkt med inputtet som: 
                \[K[n] = (2x[n] + S[n])\]
                Og så har jeg 
                \[y[n] = K[n] * 3 + 6 * S[n - 1]\]
                Eller udvidet: 
                \[y[n] = (2x[n] + S[n]) * 3 + 6 * S[n - 1]\]
                Hvis Sn egentlig bare er en tidsforsinkelse efter feedforward og feedback. 
                \[y[n] = (2x[n] + (x[n] + 1/3x[n - 1]]) * 3 + 6 * (x[n - 1] + 1/3*x[n - 2])\]
                \[y[n] = 6x[n] + 3x[n] + x[n - 1] + 6x[n - 1] + 2x[n - 2]\]
                \[=====================\]
                \[y[n] = 9x[n] + 7x[n - 1] + 2x[n - 2]\]
                \[=====================\]
                
            \end{UnderOpgave}
            \begin{UnderOpgave}[Determine the impulse response of the system]
                Ud fra min udledning så vil jeg få systemet til at være: 
                \[H(z) = \filterZto{, 7, 2}{1, -9}\]
                Men partial fraction får jeg at den kan beskrives ved: 
                \[H(z) = -0.901 + \frac{0.79}{1 - 9z^{-1}}\]
                Og det har jeg transformationer for. Det er en impuls og en faldende eksponentiel funktion. 
                Jeg antager kausalitet, så at $|z| > 1$ 
                \[h[n] = -0.901\delta[n] + 9^nu[n]\] 
                Det er så ikke en faldende eksponentiel funktion, men en voksende eksponentiel funktion. 
                Hvis ikke at jeg har udledt knudepunkterne forkert, så er det sådan, at systemet vil være. 

            \end{UnderOpgave}
        \end{Opgave}
        \begin{Opgave}[Opgave 9.2 - Sammenlign flowcharts]
            \figenoghalvfjerds{0.3}
            \begin{UnderOpgave}[Givet de to flow charts, konkluder om de repræsentere samme system]
                System a repræsentere et IIR filter. 
                System b har både feedback og feedforward. 
                \[y_a[n] = x[n] + 2r*cos(\theta)y_a[n - 1] - r^2y_a[n - 2]\]
                System b: Der er et knudepunkt med feedforward delen, jeg kalder den Sn 
                \[S[n] = x[n] + r_cos(\theta)S[n - 1] + r_sin(\theta)y_b[n - 2]\]
                \[y_b[n] = rsin(\theta)S[n] + rcos(\theta)y_b[n - 1]\]
                \[y_b[n] = rsin(\theta)(x[n] + rcos(\theta)S[n - 1] + rsin(\theta)y_b[n - 2]) + rcos(\theta)y_b[n - 1]\]
                \[y_b[n] = r*[sin(\theta)(x[n] + rcos(\theta)S[n - 1] + rsin(\theta)y_b[n - 2]) + cos(\theta)y_b[n - 1]]\]
                ... 
                Der er nogle udtryk jeg ikke kan få til at gå op. Jeg brugte transponerings metoden og konkluderede at
                \[y_b[n] = rcos(\theta) * (rsin(\theta)x[n - 1] + y_b[n - 1]) + rsinx[n] + r^2sin(y_b[n - 2])\]
                Jeg har et rcos udtryk, nogle forsinkelser og et enkelt uden og et r*r udtryk. Men jeg har også noget sinus, og jeg ved ikke om jeg har mulighed for at fjerne det på nogen måde. 
                Derfor kan jeg ikke se, at de skulle repræsentere det samme system. 

            \end{UnderOpgave}
        \end{Opgave}
        \begin{Opgave}[Opgave 9.3]  
            Fejl. y[n - m] går fra m = 1 til m = 6
            Det er rettet i ligningen nedenunder: 
            \[y[n] = 3 * \sum_{m = 0}^{5}{(1/3)^m x[n - m]} + \sum_{m = 1}^{6}{y[n - m]}\]
            Determine and draw the following structures: 
            \begin{UnderOpgave}[Direct form I ( normal )]
                Direkte form I kan beskrives ved: 
                \[s[n] = \sum_{k = 0}^{M}{b_k x[n - k]}\]
                \[y[n] = -\sum_{k=1}^{N}a_{k}y[n-k]+s[n]\]
                Som bliver til 
                \[w[n]=-\sum_{k=1}^{N}a_{k}w[n-k]+x[n]\]
                \[y[n]=\sum_{k=0}^{M}b_{k}w[n-k]\]
                I en transponering...

                \[y[n] = -\sum_{k=1}^{N}a_{k}y[n-k] + \sum_{k = 0}^{M}{b_k x[n - k]}\]
                Så med den ligning jeg har fået så kan jeg konkludere at 
                \[a = [-1, -1, ..., -1]\in\mathbb{R}^6\]
                \[b = 3 * [1, \frac{1}{3}, \frac{1}{3}^2, \frac{1}{3}^3, \frac{1}{3}^4, \frac{1}{3}^5]\]
                Der er ikke meget tilbage en at tegne
            \end{UnderOpgave}
            \begin{UnderOpgave}[Direct form II ( normal )]
                For direkte form 1 så er dens første del feedforward og dens transponerede har feedback først. 
                For direkte form 2 så er dens første del feedback og dens transponerede har feedforward først. 
                Sådan som jeg har fået den, så lægger den mest op ad en feedforward feedback opsætning. 
                Direkte form 2 har ligesom feedforward og feedback delene i trin, kan jeg også beskrive det sådan? 
                \[y[n] = 3 * \sum_{m = 0}^{5}{(1/3)^m x[n - m]} + \sum_{m = 1}^{6}{y[n - m]}\]
                \[y[n] = 3 * (\sum_{m = 0}^{5}{(1/3)^m x[n - m]} + 1/3\sum_{m = 0}^{5}{y[n - m - 1]} )\]
                \[y[n] = 3 * \sum_{m = 0}^{5}{((1/3)^m x[n - m] + 1/3y[n - m - 1])}\]
                \[y[n] = 3 * ([1x[n] + 1/3y[n - 1]] + (1/3x[n - 1] + 1/3y[n - 2]) + ((1/9)x[n - 2] + 1/3y[n - 3])\]
                \[\tab{1} + (1/27x[n - 3] + 1/3y[n - 4]) + (1/81x[n - 4] + 1/3y[n - 5]) + (1/243x[n - 5] + 1/3y[n - 6]))\] 
                Så det kan vel også deles op i pakker. Jeg konkludere at den følger den transporterede form2. Koefficienterne er de samme som for sidste opgave.
                Nedenunder så har jeg tegnet begge. 
            \end{UnderOpgave}
            \figtooghalvfjerds{0.70}
        \end{Opgave}
        \clearpage
        \begin{Opgave}[Opgave 9.8 - System strukturer: Kaskade SOS i DF]
            Givet et system 
            \[H(z) = \filterZto{1, -2.55, 4.4, 5.09, 2.41}{1, 0.26, -0.38, -0.45, 0.23}\]
            Beskriv systemet som: 
            En kaskade koblet SOS i Direkte Form I
            En kaskade koblet SOS i Transponeret Direkte Form I
            En kaskade koblet SOS i Direkte Form II
            En kaskade koblet SOS i Transponeret Direkte Form II\\

            Opgaverne gentager lidt sig selv. Jeg skal betegne og tegne systemerne.
            Men betegnelsen for sos systemet skal jeg kun gøre en gang. Derefter har jeg værdierne. 
            Jeg fulgte eksemplet i bogen hvor de bruger matlab til at finde koefficienterne $\vec{b}, \vec{a}$ for begge af de to kaskade koblede filtre.
            Deres eksempel laver den transponerede udgave af direkt form II.\\

            Med scipy.signal.tf2sos(b, a) så fandt jeg de nye b og a værdier i SOS udgaven. \\
            scipy.signal.tf2sos(b, a): 
            \[H =  
                \begin{vmatrix}
                    b_{00} & b_{01} & b_{02} & a_{00} & a_{01} & a_{02} \\
                    b_{10} & b_{11} & b_{12} & a_{10} & a_{11} & a_{12}
                \end{vmatrix}
            \]
            \figtreoghalvfjerds{0.352}   
        \end{Opgave}
        \begin{Opgave}[Opgave 9.12c - System struktur : Linær fase form]
            \[H(z) = 1 + 1.61z^{-1} + 1.74z^{-2} + 1.61z^{-3} + z^{-4}\]
            Find og tegn den lineære fase form. 
            \[Y(z) = X(z) * (1 + 1.61z^{-1} + 1.74z^{-2} + 1.61z^{-3} + z^{-4})\]
            \[y[n] = x[n] + 1.61x[n - 1] + 1.74x[n - 2] + 1.61x[n - 3] + x[n - 4]\]
            Jeg ser at den har symmetri, som jo er kriteriet for et system til at have en lineær fase. 
            For at konstruere filteret, så kan jeg omskrive differensligningen i form af dens par omkring symmetrien. 
            \[y[n] = 1 * (x[n] + x[n - 4]) + 1.61 * (x[n - 1] + x[n - 3])  + 1.74x[n - 2]\]
            Og så følger jeg bare lineær fase formen som bogen viser: 
            \figfireoghalvfjerds{0.5}
        \end{Opgave}
        \begin{Opgave}[Opgave 9.28]
            Et system er beskrevet ud fra et signal flow diagram. 
            \begin{UnderOpgave}[Bestem differens ligningen]
                Jeg finder at det nederste er et DF2 system hvor outputtet er input og output
                Derudover så er der en direkte kobling og en forsinkelse. 
                \[y[n] = x[n] + x[n - 1] + y_{df2}[n - 1]\]
                Ud fra beskrivelsen af en DF2, så har jeg udledt at 
                \[y_{df2}[n] = \frac{3}{2}y[n] + \frac{9}{8}y[n - 1]\]
                Så med det beskrevet, så kan jeg beskrive den fulde differensligning. 
                \[y[n] = x[n] + x[n - 1] + \frac{3}{2}y[n - 1] + \frac{9}{8}y[n - 2]\]
                \figfemoghalvfjerds{0.3}
            \end{UnderOpgave}
            \begin{UnderOpgave}[Bestem impuls responsen af systemet]
                \[Y(z)*(1 - \frac{3}{2}z^{-1} -\frac{9}{8}z^{-2}) = X(z)*(1 + z^{-1})\]
                \[H(z) = \frac{1 + z^{-1}}{1 - \frac{3}{2}z^{-1} -\frac{9}{8}}z^{-2}\]
                Jeg har lavet partial fraction på den: 
                \[H(z) = -\frac{0.174}{1 + 0.549z^{-1}} + \frac{1.17}{1 - 2.05z^{-1}}\]
                Og det er to eksponentielle funktioner. 
                Jeg aftager kausalitet. Med eksponentielle funktioner så skal $|z| > a$.
                Og da: 
                ROC : $|z| > 1$, $|z| > 2$
                bare er: 
                ROC : $|z| > 2$, så har jeg at systemet er stabilt for: 
                \[H(z) = -\frac{0.174}{1 + 0.549z^{-1}} + \frac{1.17}{1 - 2.05z^{-1}}, \tab{1} ROC : |z| > 2.05\]
                Så antager jeg også, at det er et LTI system: 
                \[============================\]
                \[h[n] = -0.174 * (-0.549)^nu[n] + 1.17 * (2.05)^nu[n]\]
                \[============================\]
            \end{UnderOpgave}
            Og har jeg regnet rigtigt? Jeg har intet svar og det virker ikke som om at chatten ved hvad den laver. 
            Jeg beskriv input output relationer for DF2 system i systemet, og hvordan beskrev jeg det?
            Fra formel 9.20 
            \[y[n]=\sum_{k=0}^{M}b_{k}w[n-k]\]
            Og fra formel 9.18
            \[w[n]=-\sum_{k=1}^{N}a_{k}w[n-k]+x[n]\]
            I stedet for at blande k ind begge steder, så bruger jeg k til M og l til N. 
            \[w[n - k]=-\sum_{l=1}^{N}a_{l}w[n - k - l]+x[n - k]\]
            For så kan jeg bedre kombinere udtrykkende.
            \[y_1[n]=\sum_{k=0}^{M}b_{k}(-\sum_{l=1}^{N}a_{l}w[n - k - l]+x[n - k])\]
            \[y_1[n]= - \sum_{k=0}^{M}b_{k}(\sum_{l=1}^{N}a_{l}w[n - k - l]+x[n - k])\]
            \[y_1[n]= - 0.75 * w[n - 1] - 0.75 * 0.5 * w[n - 2] + x[n] + 0.5*x[n - 1]\]
            Det er sådan jeg evaluere summationen, måske er det forkert? Grænserne er lidt mystiske måske. Der er jo ikke noget $w[n - 2]$, så det må være $w[n - 1]$
            \[y_1[n]= - 0.75 * w[n] - 0.75 * 0.5 * w[n - 1] + x[n] + 0.5*x[n - 1]\]
            Det var sådan jeg beregnede den før også.
            Jeg har lige fået wolfram til at evaluere den også: 
            \[y_1[n] = - \sum_{k=0}^{M}b_{k}(\sum_{l=1}^{N}a_{l}w[n - k - l]+x[n - k]) = -a_1b_1w[n - 2] - a_1b_0w[n - 1] - b_1x[n - 1] - b_0x[n]\]
            \[y_1[n] = -0.75*0.5w[n - 2] - 0.75*1w[n - 1] - 0.5x[n - 1] - 1 * x[n]\]
            \[y_1[n] = -0.75*0.5w[n - 2] - 0.75w[n - 1] - 1 * x[n] - 0.5x[n - 1]\]
            Og det var også sådan at jeg evaluerede den. Jeg missede dog lige nogle fortegn. 
            Det er stadigvæk mærkeligt med grænserne, for der findes kun en $w[n - 2]$. Hvis jeg bare antager at det er forkert, og at den skal rykkes en tilbage, så: 
            \[y_1[n] = - x[n] - 0.5x[n - 1] - 0.75w[n] - 0.375w[n - 1]\]
            Måske beregner jeg den faktisk forkert. Jeg beskrev jo før hvordan jeg kunne beskrive den analytisk ud fra inputtet, men det kan jeg jo egentlig ikke bare lige. 
            Men nu har jeg fundet input output relationen for den her, så lad mig prøve at evaluere den derfra. 
            \color{red} Efter at være kommet tilbage efter en uge, så vil jeg faktisk godt tro, at jeg kan beskrive det hele ud fra x og y. 
            Men hvor jeg før bare beskrev wn som yn, så vil jeg nu rette det til at det er $y[n] + 0.5y[n - 1]$
            \color{black}

        \end{Opgave}
        \begin{Opgave}[Opgave 9.29 - Udledning af differens ligninger af to systemer]
            Given 2 signal flow diagrams, describe the difference equations in terms of x and y. 
            \figseksoghalvfjerds{0.6}
            Etteren ser jeg som et kaskade second order system med to DF II systemer.\\
            I bogen bruger de et kaskade second order system med transponeret DF II, da den vidst er mere typisk brugt. 
            TDF II bruger formen. 
            \[\begin{vmatrix}
                -a_i & b_i
            \end{vmatrix}\]
            Men DF II bruger formen :
            \[\begin{vmatrix}
                b_i & -a_i
            \end{vmatrix}\]
            Og for SOS systemer, så bliver de sat i rækker fra top til bund, gående fra xn til yn. b værdierne først, så a værdierne, og ud af kolonnerne. 
            Mit sos system: 
            \[\begin{vmatrix}
                1 & 0.5 & 2 & 1 & -1/4 & -3/8 \\
                1 &  -2 & 1 & 1 &  1/3 & -2/9
            \end{vmatrix}\]

            Med scipy's sos2tf funktion så kan jeg så finde koefficienterne til den samlede filtrering.
            \[==============================\]
            \[H(z) = \filterZto{1, -3/2, -7/2, 2}{1, 1/12, -0.68, -0.069, 1/12}\]
            \[==============================\]\\\\\\ 
            
            Så til det andet filter: 
            Se jeg kan ikke helt bruge det på samme måde her. Havde det været 4 samlinger, så havde jeg kunnet beskrive den som et kaskade system af second order til et DF I filter.
            Her er der dog form af DF I men afbrudt af en DF II form i midten. Min strategi er at beskrive midten som en direkte form, og så bare udlede for siderne. 
            Op ned op for DF II er den ikke transponerede version. 
            \[y[n] = \sum_{k=0}^{2}{b_kw[n - k]}\]
            \[w[n] = \sum_{i=1}^{2}{a_iw[n - i]}\]
            Hvis jeg i stedet for at beskrive den ud fra dens beregninger i knudepunktet w, men i stedet beskriver den ud fra inputtet og dens forsinkede feedback. 
            \[w[n] = x[n] - 1/3x[n - 1] + 2/9x[n - 2]\]
            \[y[n] = \sum_{k=0}^{2}{b_k(x[n - k] - 1/3x[n - 1 - k] + 2/9x[n - 2 - k])}\]
            \[y[n] = (x[n] - 1/3x[n - 1] + 2/9x[n - 2]) + (x[n - 1] - 1/3x[n - 2] + 2/9x[n - 3]) + 4*(x[n - 2] - 1/3x[n - 3] + 2/9x[n - 4])\]
            Samler udtrykkende ud fra deres forsinkelser
            \[y[n] = x[n] + x[n - 1] * (- 1/3 + 1) + x[n - 2] * (2/9 - 1/3 + 4) + x[n - 3] * (2/9 - 4/3) + x[n - 4] * 8/9\]
            \[y[n] = x[n] + 2/3 x[n - 1] + 35/9 x[n - 2] - 10/9 x[n - 3] + 8/9 x[n - 4]\]
            På matrix form havde jeg også kunnet have beskrevet det som. \\
            $\begin{vmatrix} 1 & 1z^{-1} & 4z^{-2} \end{vmatrix} . \begin{vmatrix} x[n] \\ -1/3x[n - 1] \\ 2/9x[n - 2]\end{vmatrix}$\\
            Måske... Det havde givet bedre mening hvis feedbacket ikke havde bidraget til yderligere feedback. 
            Så jeg har midter punktet. Lade mig beskrive inputtet som outputtet til første del. 
            \[=========================================\]
            \[y_2[n] = y_1[n] + 2/3 y_1[n - 1] + 35/9 y_1[n - 2] - 10/9 y_1[n - 3] + 8/9 x[n - 4]\]
            \[=========================================\]\\

            Så til første del: 
            \[y_1 = x[n] - x[n - 1] + 0.5x[n - 2]\]
            Og sidste del: 
            \[y_3 = y_2[n] - 1/4y_2[n - 1] + 3/8y_2[n - 2]\]
            
            Måske er det nemmest at få evalueret, hvis jeg gør det med matrix metoden. 
            \[y_1 = \begin{vmatrix} 1 & -1 & 0.5 \end{vmatrix}\begin{vmatrix} x[n] \\ x[n - 1] \\ x[n - 2]\end{vmatrix}\]
            \[y_3 = \begin{vmatrix} 1 & -1/4 & 3/8 \end{vmatrix}\begin{vmatrix} y_2[n] \\ y_2[n - 1] \\ y_2[n - 2]\end{vmatrix}\]
            Og så vil jeg kunne beskrive 
            \[y_2[n] = \begin{vmatrix} 1 & 1z^{-1} & 4z^{-2} \end{vmatrix} . \begin{vmatrix} 1 \\ -1/3z^{-1} \\ 2/9z^{-1} \end{vmatrix} . \begin{vmatrix} 1 & -1 & 0.5 \end{vmatrix} . \begin{vmatrix} x[n] \\ x[n - 1] \\ x[n - 2]\end{vmatrix}\]
            Og 
            \[y_3 = \begin{vmatrix} 1 & -1/4 & 3/8 \end{vmatrix} . \begin{vmatrix} 1 \\ z^{-1} \\ z^{-2} \end{vmatrix} * \begin{vmatrix} 1 & 1z^{-1} & 4z^{-2} \end{vmatrix} . \begin{vmatrix} 1 \\ -1/3z^{-1} \\ 2/9z^{-1} \end{vmatrix} * \begin{vmatrix} 1 & -1 & 0.5 \end{vmatrix} . \begin{vmatrix} 1 \\ z^{-1} \\ z^{-2} \end{vmatrix} * x[n]\]
            Som egentlig bare giver
            \[y[n] = y_1 * y_2 * y_3\]
            Det er lidt en gråzone at skrive det sådan der. Jeg tager måske højde for konvolutionen med at skrive forsinkelser med i. I python har jeg så evalueret den til at være: 
            \[y[n] = x[n] - 5/4x[n - 1] + 0.792x[n - 2] - 0.0833x[n - 3] + 0.701x[n - 4] - 0.944x[n - 5] + 0.938x[n - 6] - 0.444x[n - 7] + 0.167x[n - 8]\]

        \end{Opgave}
        \begin{Opgave}[Opgave 9.29 - (Ny ) Udledning af differens ligninger af to systemer]
            Given 2 signal flow diagrams, describe the difference equations in terms of x and y. 
            \figseksoghalvfjerds{0.6}\\
            Grunden til at jeg har valgt at lave den om er fordi, at jeg synes at jeg kludrede lidt rundt i system 2. 
            Jeg tror måske jeg blandede z domæne og tids domæne sammen, så jeg kan ikke være sikker på, at jeg ikke har lavet fejl.\\\\
                       
            
            Etteren ser jeg stadigvæk som et kaskade second order system med to DF II systemer.\\
            TDF II bruger formen. 
            \[\begin{vmatrix}
                -a_i & b_i
            \end{vmatrix}\]
            Men DF II bruger formen :
            \[\begin{vmatrix}
                b_i & -a_i
            \end{vmatrix}\]
            Og mit system er et DF II filter.
            Mit sos system: 
            \[\begin{vmatrix}
                1 & 0.5 & 2 & 1 & -1/4 & -3/8 \\
                1 &  -2 & 1 & 1 &  1/3 & -2/9
            \end{vmatrix}\]

            Med scipy's sos2tf funktion så kan jeg så finde koefficienterne til den samlede filtrering.
            \[==============================\]
            \[H(z) = \filterZto{1, -3/2, -7/2, 2}{1, 1/12, -0.68, -0.069, 1/12}\]
            \[==============================\]\\\\\\ 
            
            Så til det andet filter. 
            Fordi der er et DF II system i midten og ikke 4 knudepunkter til et SOS på DF I form, så beskriver jeg hver blok for sig. 
            Midter blocken beskriver jeg først. Det nemmeste er at beskrive det i Z domæne fra start af. 
            I bogen har de udledt det for et DF II system: 
            \[Y(z) = H_1(z)H_2(z)X(z) = H_1(z)W(z)\]
            \[W(z) = H_2(z)X(z) = \frac{1}{1 - \sum_{k = 1}^{N}{a_kz^{-k}}}X(z)\]
            \[Y(z) = \sum_{k = 0}^{M}{b_kz^{-k}W(z)}\]
            W(z) er konstant i sidste sum og da summer af brøkker af samme nævner kan sættes sammen til en brøk, så får jeg transformationen til at være: 
            \[Y(z) = \frac{\sum_{k = 0}^{M}{b_kz^{-k}}}{1 - \sum_{k = 1}^{N}{a_kz^{-k}}}X(z)\]
            Ser jeg på konstanterne, så har jeg at 
            \[Y_2(z) = \filterZto{1, 1, 4}{1, 2/9, -1/3}X(z)\]    
            Så til ydersiderne. 
            \[y_1 = x[n] - x[n - 1] + 0.5x[n - 2] \transformation{Z} Y_1(z) = X_1(z) * (1 - z^{-1} + 0.5z^{-2})\]
            \[y_3 = x[n] + 1/4y[n - 1] + 3/8y[n - 2] \transformation{Z} Y_3(z)*(1 - 1/4z^{-1} - 3/8z^{-2}) = X_3(z)\]
            \[H(z) = H_1(z)H_2(z)H_3(z)\]
            og 
            \[Y(z) = H(z)X(z)\]
            \[Y(z) = \frac{(1 - z^{-1} + 0.5z^{-2})}{1}\filterZto{1, 1, 4}{1, 2/9, -1/3}\frac{1}{1 - 1/4z^{-1} - 3/8z^{-2}}X(z)\]
            Så kan jeg se, at jeg på en måde stadigvæk kan beskrive DF I systemet, ud fra den kun med nulpunkter og den kun med poler. Da det bare er multiplikation, så er rækkefølgen ligegyldig.
            \[Y(z) = \frac{1 - z^{-1} + 0.5z^{-2}}{1 - 1/4z^{-1} - 3/8z^{-2}}\filterZto{1, 1, 4}{1, 2/9, -1/3}X(z)\] 
            Mere simplificere jeg det ikke her. Jeg fyrer det ind i python og så lader jeg den beskrive det for mig.
            Jeg får at: 
            \[Y(z) = \filterZto{1, 0, 7/2, -7/2, 2}{1, -0.278, -0.764, 0, 1/8}X(z)\]
            Derfra har jeg så at
            \[Y(z)*(1 - 0.278z^{-1} - 0.764z^{-2} + 1/8z^{-4}) = X(z)*(1 + 7/2z^{-2} - 7/2z^(-3) + 2z^{-4})\] 
            \[y[n] - 0.278y[n - 1] - 0.764y[n - 2] + 1/8y[n - 4] = x[n] + 7/2*(x[n - 2] - x[n - 3]) + 2x[n - 4]\]
            \[=================================================\]
            \[y[n] = x[n] + 7/2*(x[n - 2] - x[n - 3]) + 2x[n - 4] + 0.278y[n - 1] + 0.764y[n - 2] - 1/8y[n - 4]\]
            \[=================================================\]
            Her har jeg beskrevet systemet ikke kun som et FIR system men også med feedback, så på den måde er den bedre beskrevet end før.
            Jeg har intet svar så jeg må gå ud fra, at det er en okay repræsentation af systemet.
        \end{Opgave}
    \end{kapitel}
    \begin{kapitel}[Design of FIR filters]
        \begin{Opgave}[Opgave 10.1 - Filter krav]
            \begin{UnderOpgave}[Givet \text{$\delta_p = 0.01, \delta_s = 0.0001$}, hvad er Ap, As, \text{$\epsilon$} og A]
                Ap er forskellen i værdierne i passbåndet, og er givet ved 
                \[A_p = A+\delta_p - (A - \delta_p) = 2\delta_p = 0.02\]
                As er givet som afstanden mellem maks værdien i passbåndet og maks værdien i stop båndet. 
                \[A_s = A + \delta_p - \delta_s = A + 0.0099\] 
                For analog filter gælder der, at 
                \[\delta_s = 1/A, \tab{0} A = 1/delta_s = 10000\]
                \[A_s = 10000 + 0.0099 \approx 10000\]
                og for analog filtre gælder der også at
                \[A = 1/\sqrt{1 + \epsilon^2}\]
                \[1 + \epsilon^2 = (1/A)^2\]
                \[\epsilon = \sqrt{(1/A)^2 - 1}\]
                \[\epsilon = \sqrt{10^{-4*2} - 1}\]
                \[\epsilon \approx \sqrt{-1} = j\]
            \end{UnderOpgave} 
            \begin{UnderOpgave}[Givet \text{$\epsilon = 0.25, A = 200$} determine de relative specifikationer Ap, As og de absolutte specifikationer \text{$\delta_p, \delta_s$}]
                Jeg fandt ud af, at jeg regnede forkert ud fra de analoge specifiktationer. 
                Formlen er lige med følgende, ikke A
                \[1 - \delta_p = 1/\sqrt{1 + (\epsilon)^2} = 1/\sqrt{1 + (1/4)^2} = (\frac{17}{16})^{-1/2}\]
                \[1 - (\frac{17}{16})^{-1/2} = \delta_p \approx 0.03 \]
                \[A_p = 2 * \delta_p = 0.06\]
                \[A_s = 1 - 1/A  = 0.995\]
                Så til næste sammenhæng: 
                \[1/A = \delta_s = 0.005\]
                Og så har jeg alle værdierne som 
                \[======\]
                \[A_p = 0.06\]
                \[A_s = 0.995\]
                \[\delta_s = 0.005\]
                \[\delta_p \approx 0.03\]
                \[======\]
            \end{UnderOpgave}   
        \end{Opgave}       
        \begin{Opgave}[Opgave 10.4 - Frekvens response mod frekvenssamplings ækvivalente]  
            For et FIR filter med impuls responsen: 
            \[H[n] = u[n] - u[n - 4]\]
            Sammenlign de praktiske værdier med dens ideele...\\
            Plots gør jeg I par. 
            \begin{UnderOpgave}[Beregn og plot magnitude responsen]
                Med fft beregner jeg dens spektrum. 
            \end{UnderOpgave}
            


            \begin{UnderOpgave}[Beregn og plot amplitude responsen og sammenlign med magnituden]
                Det er en rektangulær funktion forskudt i tiden, så det påvirker amplituderne i frekvensplottet.
                Den vil blive til en sinc funktion i frekvens spektret.
                \[A_0*rect[n/N] = \left\{ \begin{array}{ll} A_0, & |n|\geq N/2 \\ 0, & |n|>N/2\end{array}\right.\]
                \[N = 4\]
                Jeg skal så lige have rykket grænserne så at
                \[|n - n_0|<2\] 
                Gælder for $0\geq n < 4$. 
                \[|4 - n_0|\geq 2, \tab{0} n_0 = 2\]
                \[|0 - 2| \geq 2\]
                
                \[A_0*rect[\frac{n - 2}{4}] \transformation{F} A_0 \frac{sin(\omega * (N/2 + 1/2))}{sin(\omega/2)} * e^{-j\omega * 2}\] 
                \[H_id(e^{j\omega}) = \frac{sin(\omega * 5/2)}{sin(\omega/2)} * e^{-2j\omega}\]
                Nu har jeg den idéele funktion, så den vil jeg bruge til at sammenligne med den fast fourier transformerede. Både for magnituderne og for faserne.
            \end{UnderOpgave}
            \fignioghalvfjerds{0.55}\\
            Så ses der er filteret flader ud ved fft transformationen og for samme transformation, så er faserne også forsinkede.

        \end{Opgave}       
        \begin{Opgave}[Opgave 10.10]     
            Design a highpass FIR filter to satisfy the specifications 
            \[\omega_s = 0.3\pi, A_s = 50dB, \omega_p = 0.05\pi, A_p = 0.001dB\]
            \begin{UnderOpgave}[Use an appropriate fixed window to obtain a minimum length linear phase filter]
                Provide a plot similar to figure 10.12\\

                Så her skal jeg vælge et vindue som opfylder kriterierne først. 
                \[A_s = 50dB\]
                Og det kan Hamming og Blackman gøre. 
                Blackman attenuere signalet bedst men kræver flere samples.
                Hamming bliver det. 
                \[\Delta \omega = 6.6\pi / L \] 
                For hamming. 
                \[0.3 * \pi - 0.05 \pi = 6.6 \pi / (M + 1)\]
                \[0.25M + 0.25 = 6.6\]
                \[0.25M = 6.35 \]
                \[M = 25.4\]
                Er kravet. Det er ikke et helttal, og det har jeg brug for at det er. 
                Hvis jeg runder op, så vil det betyde, at transitionsbåndet vil blive mindre. Kravet er da bedre end ønsket
                Runder jeg ned så betyder det transitionsbåndet bliver breddere end ønsket.   Kravet vil da ikke være opfyldt.
                Så derfor bliver jeg nødt til at runde op. 
                \[M = 26\]
                For at sikre en linear fase, så skal jeg lave frekvens responsen til en som har linear fase.
                Det gør jeg ved at sikre symmetri. 
                En sinc funktion er symmetrisk. Den i frekvens domænet vil være en firkant. Perfekt til pass filtre.  
                







            \end{UnderOpgave}
        \end{Opgave}       
        \begin{Opgave}[Opgave 10.10 - (Nyt forsøg)]     
            Design a highpass FIR filter to satisfy the specifications 
            \[\omega_s = 0.3\pi, A_s = 50dB, \omega_p = 0.05\pi, A_p = 0.001dB\]
            \begin{UnderOpgave}[Use an appropriate fixed window to obtain a minimum length linear phase filter]
                Provide a plot similar to figure 10.12\\
                Jeg forstod ikke helt design kravene til det fulde tror jeg. Jeg har fået et ønske om et highpass filter, 
                men jeg har fået frekvenser som var det et lavpass filter.\\
                \color{teal}Strategi : \\
                Jeg vil designe den som en lavpass filter, modulere den med cos til pi og så køre et vindue på den. 
                \color{black}
                For at designe lavpass filtret, så bruger jeg discrete tid transformationsparet. 
                \[X(\omega)\,=\,\left\{\begin{array}{l l}{{1,}}&{{0\,\le\,|\omega|\,\le\,W}}\\ {{0,}}&{{W<|\omega|\,\le\,\pi}}\end{array}\right. \transformation{F^{-1}} \frac{\sin Wn}{\pi n}\]
                \[W = 2 \omega_p = 0.1\pi\]
                Jeg skal så have den frekvens forskudt med pi. 
                \[e^{j*\pi * n}x[n] \transformation{Z} X(e^{j(\omega - \pi)})\]
                Tror faktisk at det er rigeligt, fordi kun halvdelen af filtret er i enden, den anden halvdel gentager sig i starten. 
                
                Så nu ved jeg rent faktisk, hvordan jeg skal lave min sinc function. 
                Så tilbage til min gamle konklusion om vindue, antallet af samples og krav til lineær fase.
                \[A_s = 50dB\]
                Og det kan Hamming og Blackman gøre. 
                Blackman attenuere signalet bedst men kræver flere samples.
                Hamming bliver det. 
                \[\Delta \omega = 6.6\pi / L \] 
                For hamming. 
                \[0.3 * \pi - 0.05 \pi = 6.6 \pi / (M + 1)\]
                \[0.25M + 0.25 = 6.6\]
                \[0.25M = 6.35 \]
                \[M = 25.4\]
                Er kravet. Det er ikke et helttal, og det har jeg brug for at det er. 
                Hvis jeg runder op, så vil det betyde, at transitionsbåndet vil blive mindre. Kravet er da bedre end ønsket
                Runder jeg ned så betyder det transitionsbåndet bliver breddere end ønsket.   Kravet vil da ikke være opfyldt.
                Så derfor bliver jeg nødt til at runde op. 
                \[M = 26\]
                For at sikre en linear fase, så skal jeg lave frekvens responsen til en som har linear fase.
                Det gør jeg ved at sikre symmetri. 
                \[x[n] = e^{j\pi*n}\frac{sin(W*n)}{\pi n}, \tab{0} n = 0, ... 26\]
                Jeg plotter den i en sammenligning med den ideele højpass
                \figfirs{0.4}
                \figenogfirs{0.4}




            \end{UnderOpgave}
        \end{Opgave}
        \begin{Opgave}[Opgave 10.10 - (Nyt nyt forsøg)]     
            Design a highpass FIR filter to satisfy the specifications 
            \[\omega_s = 0.3\pi, A_s = 50dB, \omega_p = 0.05\pi, A_p = 0.001dB\]
            \begin{UnderOpgave}[Use an appropriate fixed window to obtain a minimum length linear phase filter]
                Provide a plot similar to figure 10.12\\
                Det lykkedes mig ikke at lave filteren rigtig i forhold til numpy. Nu forstår jeg det lidt bedre, så jeg prøver på ny. 
                Strategien er den samme.
                \color{teal}Strategi : \\
                Jeg vil designe den som en lavpass filter, frekvens forskyd den med pi, og så køre et vindue på den. 
                \color{black}
                For at designe lavpass filtret, så bruger jeg discrete tid transformationsparet. 
                \[X(\omega)\,=\,\left\{\begin{array}{l l}{{1,}}&{{0\,\le\,|\omega|\,\le\,W}}\\ {{0,}}&{{W<|\omega|\,\le\,\pi}}\end{array}\right. \transformation{F^({-1})} \frac{\sin Wn}{\pi n}\]
                \[W = 2 \omega_p = 0.1\pi\]
                Jeg skal så have den frekvens forskudt med pi. 
                \[e^{j*\pi * n}x[n] \transformation{Z} X(e^{j(\omega - \pi)})\]
                Tror faktisk at det er rigeligt, fordi kun halvdelen af filtret er i enden, den anden halvdel gentager sig i starten. 
                
                Så nu ved jeg rent faktisk, hvordan jeg skal lave min sinc function. 
                Så tilbage til min gamle konklusion om vindue, antallet af samples og krav til lineær fase.
                \[A_s = 50dB\]
                Og det kan Hamming og Blackman gøre. 
                Blackman attenuere signalet bedst men kræver flere samples.
                Hamming bliver det. 
                \[\Delta \omega = 6.6\pi / L \] 
                For hamming. 
                \[0.3 * \pi - 0.05 \pi = 6.6 \pi / (M + 1)\]
                \[0.25M + 0.25 = 6.6\]
                \[0.25M = 6.35 \]
                \[M = 25.4\]
                Er kravet. Det er ikke et helttal, og det har jeg brug for at det er. 
                Hvis jeg runder op, så vil det betyde, at transitionsbåndet vil blive mindre. Kravet er da bedre end ønsket
                Runder jeg ned så betyder det transitionsbåndet bliver breddere end ønsket.   Kravet vil da ikke være opfyldt.
                Så derfor bliver jeg nødt til at runde op. 
                \[M = 26\]
                For at sikre en linear fase, så skal jeg lave frekvens responsen til en som har linear fase.
                Det gør jeg ved at sikre symmetri. 
                \[x[n] = e^{j\pi*n}\frac{\sin Wn}{\pi n}, \tab{0} n = 0, ... 26\]
                Jeg plotter den i en sammenligning med den ideele højpass
                \figtoogfirs{0.4}
                \figtreogfirs{0.4}
                På grund af de det lave antal samples, så har det medført, at toppen er blevet fladet ud i vindue processen. 
                Den har dæmpet stopbåndet meget, men ja det kostede på passbåndet. 
                Her kan jeg gå to veje. \\
                1. Filteret rammer $w_p$ i 0.03 pi ca. Jeg kunne lave filteret snævre får så at bredde spektret lidt. 
                Den metode vil gøre designet mere praktisk og behøves ikke flere samples. 
                2. Forhøje antallet af samples. Jo flere samples, jo tættere kommer filterets frekvensspektrum mod en rectangle med kanter i wp. \\
                Jeg har et rimeligt stort gab for i transitions båndet, så jeg behøves ikke nødvendigvis flere samples.
                Jeg har i stedet for designet for en $w_p = 0.15\pi$
                \figfireogfirs{0.4}
                Fejlen ses betydeligt mindre med det design.
            \end{UnderOpgave}
            \begin{UnderOpgave}[Brug en førlavet funktion]
                Jeg har brugt scipy.signal.firwin til at lave: 
                \figfemogfirs{0.4}\\
                Den skal tages med et gran salt. Det er ikke lige så nemt at vise den analytisk.
                Den bruger antallet af samples som resolution og ikke andet.\\
                Nu har jeg brugt langt tid på at finde ud af hvordan jeg selv kunne lave den og ikke meget tid på den her. 
                Måske er den bedre til at designe det, men lige nu, så vil jeg bruge min. 
                Matlabs funktioner er egentlig relativ nem: \\
                \figseksogfirs{0.4}
                Og den opfylder kravene meget godt. 
                Den beregner det lidt på en anden måde som det ses i intervallet. Derudover så har jeg også fundet ud at forøge dens resolution, det har jeg ikke for scipys.\\
            \end{UnderOpgave}
            Jeg har i den her opgave vist forstærkningen bare. Det gjorde jeg bare for at se om det hele virkede. Næste skridt vil så være, at se det i dB. \\
            For lige nu har jeg ikke set om kravene er opfyldt. Men nok om den opgave for nu 
        \end{Opgave}
        \begin{Opgave}[Opgave 10.11 - Consider an ideal lowpass filter with cutoff frequency \text{$w_c = \pi/2$}]
            \begin{UnderOpgave}[Using L = 20 samples around the unit circle, compute the resulting impulse response hn using the rectangular window.]
                Compute and plot the magnitude response over $0 \leq \omega \leq 2\pi$ and show the frequency samples on the magnitude response plot. 
                Så det er altså en opgave om frequency sampling. 
                Der er forskellige formler for Type I, type II og Type III, type IV. Jeg har konstrueret det før for type II. 
                Med min opsætning får jeg at hn har lige symmetri, og $M = L - 1 = 19$ som er ulige. Så her er også tale om et type II filter. \\

                Metoden: 
                Trin 1: Placer ønskede amplituder til deres respektative frekvens.
                \begin{verbatim}
                    L = 20
                    alpha = np.int64(L/2)
                    w_k = np.linspace(0, np.pi, alpha)
                    Ad = np.zeros(alpha)                    # Frequency values
                    idxlp = np.where(w_k <= np.pi/2)        # Lp indices 
                    Ad[idxlp] = 1
                    Ad = np.hstack([Ad, 0, Ad[::-1][:9]])   # -pi : 0 == pi : 2pi
                \end{verbatim}
                Hvor jeg laver et lavpass filter med symmetri i n = 0. Den sidste halvdel af Ad er det nederste interval $[-\pi: 0]$. \\
                
                Trin 2: Opskriv faseren som sikre at amplituden passer til dens frekvens. 
                \begin{verbatim}
                    Q = np.int64(floor((L - 1)/2))
                    C = 2*np.pi/L 
                    kid = np.hstack([np.arange(Q + 1), 
                                     np.arange(-(Q + 1), 0, 1)]) # Index manipulation for k
                    # k = 0, ...Q; -(L - k) = -20 + 10, ... -20 + 19 
                    psidk = - Q * C * kid                        # q = 0 (type I, type II)
                    H_oensket = Ad * np.exp(1j*psidk)
                \end{verbatim}
                Trin 3: Inverse fourier transform. \\
                Få impuls responsen. 

                Trin 4: Brug vindue til at koncentrere energi. \\
                Her er valget så bare stødt på et rektangulær vindue, som efterlader alle responser uændret. \\
                Så det er som havde der ingenting sket, men det kan godt være, at det har nogle fysiske egenskaber. \\

                Trin 5: Fourier transform
                Her fourier transformere man så det nye forbedrede filter. Da den ikke er blevet forbedret, 
                så er den lige med hvad den var inden transformationen. 
                Det er til gengæld i den her process, at jeg analytisk kan oppe resolutionen i spektret ved at padde. \\
                Mit resultat
                \figethundredeogseks{0.32}
            \end{UnderOpgave}
            \begin{UnderOpgave}[Brug L = 400 og sammenlign med tidligere resultater]
                \figethundredeogsyv{0.35}
                Der ses tydeligt, at frekvensspektret konvergerer mod dens reelle værdi. 
                Der ses også hvad jeg har sat af krav for frekvens sampling. 
                Jeg har ønsket at frekvens samplingen skulle kunne beskrives symmetrisk, og det har betydet, at for L = 20 er der store skridt pr sample, 
                så mit passbånd starter først sent nær 2 pi. 
                Med mange samples betyder det ikke meget om jeg vælger frekvensen før eller efter, de er alle sammen tæt på hinanden. 
            \end{UnderOpgave}
            \begin{UnderOpgave}[Gengiv opgave b, men med hamming som vindue]
                \figethundredeogotte{0.6}
                Her har vinduen fået lov til at koncentrere energien om de frekvenser der er vigtige.
                Der ses at hvor der før var et oversving ved transitionen, så er den nu næsten ideel. 
            \end{UnderOpgave}
        \end{Opgave}
        \begin{Opgave}[Opgave 10.13 - Chebyshev polynomials]
            \begin{UnderOpgave}[Using the trigonometric identity cos(A + B) = cos(A)cos(B) - sin(A)sin(B) show that]
                \[cos[(n + 1)\omega] = 2cos(\omega)cos(n\omega) - cos[(n - 1)\omega], n \geq 1 \]
                \[cos(n\omega + \omega) = cos(n\omega)cos(\omega) - sin(n\omega)(-1sin(-\omega))\]
                \[cos(n\omega + \omega) = cos(n\omega)cos(\omega) + sin(n\omega)(sin(-\omega))\]
                Så bruger jeg identiteten og ser hvad sin kan beskrives som
                \[sin(n\omega)(sin(-\omega)) = cos(n\omega)cos(-\omega) - cos[(n - 1)\omega]\]
                Substitutere
                \[cos(n\omega + \omega) = cos(n\omega)cos(\omega) + (cos(n\omega)cos(-\omega) - cos[(n - 1)\omega])\]
                cos er den samme til en positive frekvens som til en negativ frekvens.
                \[cos(n\omega + \omega) = cos(n\omega)cos(\omega) + (cos(n\omega)cos(\omega) - cos[(n - 1)\omega])\]
                \[cos(n\omega + \omega) = 2cos(n\omega)cos(\omega) - cos[(n - 1)\omega]\]
                Og så har jeg udledt den. 3
            \end{UnderOpgave}
            \begin{UnderOpgave}[Use the formula in (a) to derive the following recursive formula]
                \[T_{n + 1} = 2xT_n(x) - T_{n - 1}(x), n\geq 1\]
                \[cos(n\omega + \omega) = 2cos(n\omega)cos(\omega) - cos[(n - 1)\omega]\]
                Jeg sætter $T_n(x) = cos(n\omega)$, hvor $x = cos(\omega)$
                Substituere for dens udtryk. 
                \[T_{n + 1} = 2xT_n(x) - T_{n - 1}(x), n\geq 1\]
            \end{UnderOpgave}
            \begin{UnderOpgave}[Use the recursion in (b) to derive the first five Chebyshev polynomials.]
                \[T_{2} = 2xT_n(x) - T_{n - 1}(x), n\geq 1\]
                Scriptet \\
                \begin{verbatim}
                def chebyshev(self, n, w, Tn): \\
                    if n < 1 : return 1 \\
                    poly = 2*cos(w)*self.chebyshev(n-1, w, Tn) - self.chebyshev(n-2, w, Tn)\\
                    if n > len(Tn): \\
                        Tn += [poly] \\
                    return poly \\
                \end{verbatim}
                klarer mosten, sympy reducere så udtrykket med simplificeringer, derfor det mulighvis ser meget forenklet ud.\\
                Det plejer at være rigtigt. 
                De første værdier sagde chatten typisk sættes til: 
                \[T_0(x) = 1\]
                \[T_1(x) = cos(\omega)\]\\
                \[T_{2}{\left(x \right)} = \cos{\left(2 \omega \right)}\]
                \[T_{3}{\left(x \right)} = \cos{\left(3 \omega \right)}\]
                \[T_{4}{\left(x \right)} = 8 \cos^{4}{\left(\omega \right)} - 8 \cos^{2}{\left(\omega \right)} + 1\]
                \[T_{5}{\left(x \right)} = \left(16 \cos^{4}{\left(\omega \right)} - 20 \cos^{2}{\left(\omega \right)} + 5\right) \cos{\left(\omega \right)}\]
                \[T_{6}{\left(x \right)} = 32 \cos^{6}{\left(\omega \right)} - 48 \cos^{4}{\left(\omega \right)} + 18 \cos^{2}{\left(\omega \right)} - 1\]
            \end{UnderOpgave}
            Jeg troede faktisk, at jeg ville lære at bruge chebyshev her. Det gjorde jeg så ikke. \\
            Men altså Henrik valgte ikke selv en opgave med den, så måske er det ikke så vigtigt igen. 

        \end{Opgave}
        \begin{Opgave}[Opgave 10.14 - Fejl estimering]
            Consider the polynomial $f(x) = 1—2x+4x^2 2x^3$. 
            We want to approximate it using a second-order polynomial 
            \[P_2(x) = a_0 +a_1x+a_2x^2\]
            so that the error 
            \[e(x) \colon f(x) — P_2(x)\]
            is equiripple over $0\leq x \leq 1$.
            \begin{UnderOpgave}[Choose an initial set of m + 2 = 4 nodes \text{$\left[\zeta_{k}\right]_{k=0}^{3}$} as 0, 1/3, 2/3, and 1, ...]
                and solve for coefficients $[a_l]_{l = 0}^2$ and $\delta$ using (10.110). 
                Graph f(x), the resulting P2(x), and e(x) in one plot.\\
                Formel 10.100: 
                \[e(\zeta_{k})=f(\zeta_{k})-P_{m}(\zeta_{k})=(-1)^{k}\delta,\quad k=0,1 m+1\]
                Equiripple: "Alternating maxima and minima all have the same amplitude". 
                Af hvad jeg forstår det, så lægger der ikke implicit, at det er et stop- eller passbånd.\\
                Det fortæller hellere ikke om der i intervallet sker en transition fra stopbånd til passbånd, eller omvendt.\\
                Det den fortæller mig om er, at i det her interval, så kan jeg beskrive mit filter ud fra $\text{ideel filter} \pm \delta$. 
                Så det er bare fejlen i det her interval. \\
                Det her er først og fremmest et lineært system af ligninger som skal løses for. 
                \[P = [a_0, a_1, a_2, a_3],\]
                \[f = [1, -2, 4, -2]\]
                Og
                \[x = 
                \begin{vmatrix}
                    x^0 \\
                    x^1 \\
                    x^2 \\
                    x^3 
                \end{vmatrix}\]
                Så 
                \[e_x = f.x - P.x\]
                For alle værdierne. 
                \[x = 
                \begin{vmatrix}
                    x_0^0 & x_1^0 & x_2^0 & x_3^0 \\
                    x_0^1 & x_1^1 & x_2^1 & x_3^1 \\
                    x_0^2 & x_1^2 & x_2^2 & x_3^2 \\
                    x_0^3 & x_1^3 & x_2^3 & x_3^3
                \end{vmatrix}\]
                For ved række søjleproduker vil jeg få. 
                \[P.x = 
                \begin{vmatrix}
                    a_0x_0^0 + a_1x_0^1 + a_2x_0^2 + a_3x_0^3 \\
                    ...
                    a_0x_3^0 + a_1x_3^1 + a_2x_3^2 + a_3x_3^3
                \end{vmatrix}\]
                Og hvis jeg gør det for både P.x og f.x så får jeg fejlen til de forskellige punkter ned ad rækkerne. Og dem har jeg allerede. 
                \[{e_x}\vec{x} = \vec{f}.\vec{x} - \vec{P}.\vec{x}\]
                \[\vec{P}.\vec{x} = \vec{f}.\vec{x} - {e_x}\vec{x}\]
                Nevermind. Jeg kender ikke fejlen endnu, for jeg kender ikke ripplen. 
                Jeg kommer også lige til at se at P kun tager 3x, og det giver jo mening, eftersom at jeg også skal løse for delta.
                I en matrix vector multiplikation som løser: 
                \[A.x = b\]
                Så er koefficienterne hen ad rækkerne i A. 
                og deres værdier er at finde i x. 
                Det gør det lidt forvirrende, at min matrix også hedder X, så lad mig kalde den A i stedet for, så jeg følger konventionen. 
                \[A = 
                \begin{vmatrix}
                    x_0^0 & x_1^0 & x_2^0 & (-1)^0 \\
                    x_0^1 & x_1^1 & x_2^1 & (-1)^1 \\
                    x_0^2 & x_1^2 & x_2^2 & (-1)^2 \\
                    x_0^3 & x_1^3 & x_2^3 & (-1)^3
                \end{vmatrix}\]
                \[x = 
                \begin{vmatrix}
                    a_0 \\
                    a_1 \\
                    a_2 \\
                    \delta 
                \end{vmatrix}\]
                og 
                \[b = 
                \begin{vmatrix}
                    f(x_1) \\
                    f(x_2) \\
                    f(x_3) \\
                    f(x_4) 
                \end{vmatrix}\]
                Jeg løser for systemet og får at 
                \[a_0 \approx 3, \tab{0} a_1 = -5.5, \tab{0} a_2 = 3.6, \tab{0} \delta \approx -0.137\]
                \figethundredeogni{0.7}
            \end{UnderOpgave}
            \begin{UnderOpgave}[Vælg 4 punkter og gentag a og plot, indtil at equiripple er fundet i det fulde interval]
                Fejlene er store i intervallet $0 \leq x \leq 0.6$ og størst omkring 0. 
                Jeg vælger 
                \[x = [0.1, 0,2, 0.3, 0.5]\]
                
            \end{UnderOpgave}
        \end{Opgave}
        \begin{Opgave}[Opgave 10.25 - Hann vinduet bedre forstået]
            \[w[n]=[0.5-0.5\cos(2\pi n/M)]w_{\mathrm{R}}[n].\]     
            $w_R[n]$ er rektangel funktionen
            \begin{UnderOpgave}[Express the DTFT of w in terms of the DTFT of wR]
                Så mig prøve at beskrive det. 
                Det er en modulering og derfor. 
                \[x[n]y[n] = \int_{2\pi}{X(e^{j\theta}) Y(e^{j (\omega - \theta)})}\]
                Så det er egentlig bare convolution. 
                For xn delen her, så vil jeg bare have 3 impulser, så regnestykket kan beskrives som en simpel sum. 
                Mine beregninger tager et spektrum in mente. Men selvfølgelig gentager det sig for hver 2 pi. 
                \[x[n] \transformation{F} 0.5 * 2\pi * \delta(\omega) - 0.5 * \pi * (\delta(\omega - 2\pi/M) + \delta(\omega + 2\pi/M))\]
                Og for convolution med impulser, så bliver spektret bare flyttet til den frekvens.
                \[W(e^{j\omega}) = \frac{1}{2\pi} * (\pi W_R(e^{j\omega}) - 0.5 * \pi * (W_R(e^{j(\omega - 2\pi/M)}) + W_R(e^{j(\omega + 2\pi/M)})))\]
                \[W(e^{j\omega}) = 0.5W_R(e^{j\omega}) - 1/4 * (W_R(e^{j(\omega - 2\pi/M)}) + W_R(e^{j(\omega + 2\pi/M)}))\]                
            \end{UnderOpgave}
            \begin{UnderOpgave}[Explain why the Hann window has the wider mainlobe but lower sidelobes than the rectangular window of the same length]
                Ud fra hvordan den er lavet, som jeg lige har vist, så har siderne en faktor 1/2 til midterspektret. Det kan så være at sidespektrene bliver lige så store som midterspektret, pga. manglende resolution i samplingen. 
                Her vil det være for M = 2, hvor: 
                \[W(e^{j\omega}) = 0.5W_R(e^{j\omega}) - 1/4 * (W_R(e^{j(\omega - \pi)}) + W_R(e^{j(\omega + \pi)}))\]                
                Og pga. gentagelserne i spektrene. 
                \[W(e^{j\omega}) = 0.5W_R(e^{j\omega}) - 2/4 * (W_R(e^{j(\omega - \pi)}))\]                
                Men udover for det enkelte special tilfælde, så vil midterspektret være større end sidespektrene. 
            \end{UnderOpgave}
        \end{Opgave}       
    \end{kapitel}
\end{Opgaver}
