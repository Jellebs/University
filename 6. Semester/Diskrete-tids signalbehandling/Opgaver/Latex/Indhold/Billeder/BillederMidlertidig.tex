\graphicspath{{Indhold/Billeder/}}
% Indstillinger: 
% Billednr:46

\newcommand{\fetchBillede}[2]{\includegraphics[scale=#2]{#1}}
\newcommand{\fetchSVG}[2]{\includesvg[scale=#2]{#1}} % width=1.0\textwidth
\newcommand{\fig}[2]{
    \begin{figure}[h!]
        % \centering
        % \top
        \makebox[\textwidth]{
            \fetchBillede{#1}{#2}
        }
        \caption{#1}
        \label{fig:#1}
    \end{figure}
}



\newcommand{\figSVG}[2]{
    \begin{figure}[h!]
        \centering
        \makebox[\textwidth]{
            \fetchSVG{#1}{#2}
        }
        \caption{#1}
        \label{fig:#1}
    \end{figure}
}


% \figen{0.2} => scaling = 0.2
\newcommand{\figen}[1]{\fig{Z transformations egenskaber}{#1}}
\newcommand{\figto}[1]{\fig{Blokdiagram kapitel 2.png}{#1}}
\newcommand{\figtre}[1]{\figSVG{Feedback system.svg}{#1}}
\newcommand{\figfire}[1]{\fig{Opgave 2.1.png}{#1}}
\newcommand{\figfem}[1]{\fig{Opgave 2.3.png}{#1}}
\newcommand{\figseks}[1]{\fig{Opgave 2.5.png}{#1}}
\newcommand{\figsyv}[1]{\fig{Opgave til kapitel 2.png}{#1}}
\newcommand{\figotte}[1]{\fig{Opgave 2.17.a.png}{#1}}
\newcommand{\figni}[1]{\fig{Opgave 2.17.b.png}{#1}}
\newcommand{\figti}[1]{\fig{Opgave 2.17.c.png}{#1}}
\newcommand{\figeleve}[1]{\fig{Opgave 2.17.png}{#1}}
\newcommand{\figtolv}[1]{\fig{Opgave 2.33.png}{#1}}
\newcommand{\figtretten}[1]{\figSVG{Opgave 2.50.svg}{#1}}
\newcommand{\figfjorten}[1]{\fig{Opgave 2.50.png}{#1}}
\newcommand{\figfemten}[1]{\fig{Opgave 3.1.a.png}{#1}}
\newcommand{\figseksten}[1]{\fig{Opgave 3.1.b.png}{#1}}
\newcommand{\figsytten}[1]{\fig{Opgave 3.1.d.png}{#1}}
\newcommand{\figatten}[1]{\fig{SløretBillede.png}{#1}}
\newcommand{\fignitten}[1]{\fig{AfsløretBillede.png}{#1}}
\newcommand{\figtyve}[1]{\fig{AfsløretsFilter.png}{#1}}
\newcommand{\figenogtyve}[1]{\fig{Opgave 3.15.png}{#1}}
\newcommand{\figtoogtyve}[1]{\fig{Opgave 3.16.png}{#1}}
\newcommand{\figtreogtyve}[1]{\fig{Opgave 3.19 - pzmap.png}{#1}}
\newcommand{\figfireogtyve}[1]{\fig{Opgave 3.19 - Respons.png}{#1}}
\newcommand{\figfemogtyve}[1]{\fig{Opgave 5.2.png}{#1}}
\newcommand{\figseksogtyve}[1]{\fig{Opgave 5.2.e.png}{#1}}
\newcommand{\figsyvogtyve}[1]{\fig{Opgave 5.30.png}{#1}}
\newcommand{\figotteogtyve}[1]{\fig{Opgave 5.48.png}{#1}}
\newcommand{\figniogtyve}[1]{\fig{Opgave 5.48.2.png}{#1}}
\newcommand{\figtredive}[1]{\fig{EKGdata.png}{#1}}
\newcommand{\figenogtredive}[1]{\fig{EKGfrekvenspektrum.png}{#1}}
\newcommand{\figtoogtredive}[1]{\fig{EKGnotchfilterAnalyse.png}{#1}}
\newcommand{\figtreogtredive}[1]{\fig{EKGnotchfilterPZmap.png}{#1}}
\newcommand{\figfireogtredive}[1]{\fig{EKGsignalfiltreret.png}{#1}}
\newcommand{\figfemogtredive}[1]{\fig{EKGforbedretNotchfilterAnalyse.png}{#1}}
\newcommand{\figseksogtredive}[1]{\fig{EKGforbedretNotchfilterPZmap.png}{#1}}
\newcommand{\figsyvogtredive}[1]{\fig{EKGforbedretSignalFiltrering.png}{#1}}
\newcommand{\figotteogtredive}[1]{\fig{SignalbehandlingTemperaturdata.png}{#1}}
\newcommand{\figniogtredive}[1]{\fig{SignalbehandlingTemperaturdataFiltreret.png}{#1}}
\newcommand{\figfyrre}[1]{\fig{SignalbehandlingTemperaturdataFilteranalyse.png}{#1}}
\newcommand{\figenogfyrre}[1]{\fig{SignalbehandlingTemperaturdataIIRFiltermodFIRFilter.png}{#1}}
\newcommand{\figtoogfyrre}[1]{\fig{Signalbehandling3DAudioelev50H50e096a.png}{#1}}
\newcommand{\figtreogfyrre}[1]{\fig{FilterStabilitet.png}{#1}}
\newcommand{\figfireogfyrre}[1]{\fig{Opgave 5.16.png}{#1}}
\newcommand{\figfemogfyrre}[1]{\fig{Opgave 5.31b.png}{#1}}
\newcommand{\figseksogfyrre}[1]{\fig{Opgave 5.31c.png}{#1}}
