\graphicspath{{Indhold/Billeder/}}
% Indstillinger: 
% Billednr:15

\newcommand{\fetchBillede}[2]{\includegraphics[scale=#2]{#1}}
\newcommand{\fetchSVG}[2]{\includesvg[scale=#2]{#1}} % width=1.0\textwidth
\newcommand{\fig}[2]{
    \begin{figure}[h]
        \fetchBillede{#1}{#2}
        \caption{#1}
        \label{fig:#1}
    \end{figure}
}
\newcommand{\figSVG}[2]{
    \begin{figure}[h]
        \fetchSVG{#1}{#2}
        \caption{#1}
        \label{fig:#1}
    \end{figure}
}

% \figen{0.2} => scaling = 0.2
\newcommand{\figen}[1]{\fig{Blokdiagram kapitel 2.png}{#1}}
\newcommand{\figto}[1]{\figSVG{Feedback system.svg}{#1}}
\newcommand{\figtre}[1]{\fig{Opgave 2.1.png}{#1}}
\newcommand{\figfire}[1]{\fig{Opgave 2.3.png}{#1}}
\newcommand{\figfem}[1]{\fig{Opgave 2.5.png}{#1}}
\newcommand{\figseks}[1]{\fig{Opgave til kapitel 2.png}{#1}}
\newcommand{\figsyv}[1]{\fig{Opgave 2.17.a.png}{#1}}
\newcommand{\figotte}[1]{\fig{Opgave 2.17.b.png}{#1}}
\newcommand{\figni}[1]{\fig{Opgave 2.17.c.png}{#1}}
\newcommand{\figti}[1]{\fig{Opgave 2.17.png}{#1}}
\newcommand{\figeleve}[1]{\fig{Opgave 2.33.png}{#1}}
\newcommand{\figtolv}[1]{\figSVG{Opgave 2.50.svg}{#1}}
\newcommand{\figtretten}[1]{\fig{Opgave 2.50.png}{#1}}
\newcommand{\figfjorten}[1]{\fig{Opgave 3.1.a.png}{#1}}
\newcommand{\figfemten}[1]{\fig{Opgave 3.1.b.png}{#1}}

