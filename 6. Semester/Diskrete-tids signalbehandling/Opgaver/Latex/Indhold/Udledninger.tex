\begin{Udledninger}
    \begin{underrubrik}[Fra convolution til differens ligning]
        Antaget at responsen kan beskriveses som et eksponentiel impuls respons.
        \[h[n] = ba^nu[n],\tab{0} -1<a<1\]
        \[y[n] = x[n] * h[n] = bx[n] + bax[n-1] + ba^2x[n-2] + ... + ba^Nx[n-N]\]
        \[y[n] = x[n] * h[n] = bx[n] + a*(bx[n-1] + bax[n-2] + ... + ba^{N-1}x[n-N])\]
        \[y[n] = x[n] * h[n] = bx[n] + a*y[n-1]\]
        Pa den måde har jeg gået fra en ligning med potentiel krav på uendelig hukommelse,
        til et system hvor man kun skal kende det tidligere output. 
        \figto{1}\\
    \end{underrubrik}
    \begin{underrubrik}[Z transformation - Kompleks konjugerede poler]
        En egenskab man kan bruge, når polerne er kompleks konjugerede. 
        Givet eksemplet. 
        \[X(z) = \filterZ{1, 1}{1, -1, 1/2}, \tab{0} p = \frac{1}{\sqrt{2}} * e^{\pm j\pi/4}\]
        Der ses så, at polerne er kompleks konjugerede
        Så laves der partial fraction på den
        \[X(z) = \filterZ{1, 1}{1, -1, 1/2} = \frac{A_1}{1 - p_1z^-1} + \frac{A_2}{1 - p_2z^-1}\]
        Og fra den fåes ligningen: 
        \[z + 1 = A_1 * (z - p_2) + A_2 * (z - p_1)\]
        Vi kan udregne, at koefficienterne A1 og A2 også skal være hinandens kompleks konjugerede.
    
        Da den Z transfomerede kunne beskrives som to simple funktioner får jeg, med koefficienter ganget på. 
        Linearitets princippet antages at være gældende her. 
        \[x[n] = A_1*(p_1)^n*u[n] + A_1^\star * (p_1\star)^n*u[n]\]
        Udvidet til eksponentiel form \[A_1 = Ae^{j\omega}, \quad p_1 = re^{j\omega_0}\]
        \[x[n] = Ar^n * (e^{j\omega_0*n} * e^{j\theta} + e^{-j\omega_0*n} * e^{-j\theta})u[n]\]
        Og da jeg ved at 
        \[cos(\theta) = \frac{e^{j\theta} + e^{-j\theta}}{2}\]
        \[x[n] = 2 * Ar^n * \cos(\omega_0 * n + \theta)u[n]\]
        Og hvis jeg husker hvordan jeg har beskrevet A, theta, omega0 og r
        \[x[n] = \frac{\sqrt{10}}{\sqrt{2}} * cos(\frac{\pi}{4} -71.56^o)u[n] = \sqrt{5} * cos(\frac{\pi}{4} -71.56^o)u[n]\]
        Så ved at indse, at der var kompleks konjugerede poler, så kunne jeg have indset, at det skulle have været en harmonisk funktion
    \end{underrubrik}
    \begin{underrubrik}[Z transformation - Kompleks konjugerede poler og koefficienter]
        Fra opgave 5.30
        \[H(z) = 10 + a * \frac{1}{1 - bz^{-1}} + a^\star * \frac{1}{1 - b^\star z^{-1}} + a^\star * \frac{1}{1 - bz^{-1}} + a * \frac{1}{1 - b^\star z^{-1}}\]
        Jeg ser en symmetri. 
        \[a^nu[n] \transformation{Z} \frac{1}{1 - a*z^{-1}}\]
        \[a^nu[n] + (a^\star)^nu[n] = (re^{j\theta})^nu[n] + (re^{-j\theta})^nu[n]\]
        \[\tab{6}                   = r^nu[n](e^{j\theta})^n + r^nu[n](re^{-j\theta})^n\]
        \[\tab{6}                   = (r^nu[n])*((e^{j\theta})^n + (e^{-j\theta})^n)\]
        \[\tab{6}                   = (r^nu[n])*(e^{j\theta n} + e^{-j\theta n})\]
        \[\tab{6}                   = (r^nu[n])* 2 * cos(\theta n)\]
        Så jeg har to filtre der kan beskrives på denne måde. 
        \[b = (0.375 + 0.65i)\]
        \pgfmathparse{sqrt((0.375)^2 + (0.65)^2)} \edef\radius{\pgfmathresult}
        \pgfmathparse{atan(0.65/0.375)} \edef\vinkel{\pgfmathresult}
        \[r_b = \radius\]
        \[\theta_b = \vinkel\] 
        Og den er den samme for alle af dem. 
        \[H(z) = 10 + a * r_b^nu[n] * 2cos[\frac{\pi}{3} * n] + a^\star r_b^nu[n] * 2cos[\frac{\pi}{3} * n]\]
        Splitter a op i dens komponenter. 
        \pgfmathparse{sqrt((2.25)^2 + (0.674)^2)} \edef\radius{\pgfmathresult}
        \pgfmathparse{180 + atan(0.674/(-2.25))} \edef\vinkel{\pgfmathresult} % Anden kvadrant
        \[r_a = \radius\]
        \[\theta_a = \vinkel\] 
        \[H(z) = 10 + (r_a e^{j\theta_a}) * r_b^nu[n] * 2cos[\frac{\pi}{3} * n] + (r_a e^{-j\theta_a}) * r_b^nu[n] * 2cos[\frac{\pi}{3} * n]\]
        \[H(z) = 10 + r_a r_b^nu[n] * 2cos[\frac{\pi}{3}*n] * (e^{j\theta_a} + e^{-j\theta_a})\]
        \[H(z) = 10 + 2 * cos(\theta_a) r_a r_b^nu[n] * 2cos[\frac{\pi}{3}*n]\]
        Så jeg har at for et filter med formen 
        \[H(z) = a * (\frac{1}{1 - bz^{-1}} + \frac{1}{1 - b^\star z^{-1}}) + a^\star * ( \frac{1}{1 - b^\star z^{-1}} + \frac{1}{1 - bz^{-1}})\]
        Så kan jeg simplificere den til at være. 
        \[==================\]
        \[h[n] = 4 cos(\theta_a) r_a r_b^nu[n] cos[\theta_b n]\]
        \[==================\]
    \end{underrubrik}



    \begin{underrubrik}[Z transformation - Egenskaber]
        Den kommukative egenskab:
        \[\convolution{x}{h}\]
        Ny variabel $l = n - k, \tab{0} k = n - l$
        Så hvad er mine nye grænser? 
        Når $k=-\infty, \tab{0} l = \infty$ stortset, for n ikke uendelig.
        Når $k=\infty, \tab{0} l = -\infty$ stortset, for n ikke uendelig.
        
        \[y[n] = \sum_{l = \infty}^{-\infty}{x[n - l]*h[l]}\]
        Jeg summer over alle de negative værdier sidst, først over de positive. 
        At gøre det den modsatte vej gør ingenforskel da: $a + b = b + a$
        \[y[n] = \sum_{l = -\infty}^{\infty}{x[n - l]*h[l]}\]
        \[y[n] = \sum_{l = -\infty}^{\infty}{h[l]*x[n - l]}\]
        Som følger samme form som
        \[\convolution{h}{x}\]
        Og er derfor det samme som
        \[\convolution{x}{h}\]
    \end{underrubrik}
    \begin{Udklip}
        \begin{underrubrik}[Z transformation - Egenfunktioner]
            Motivationen for at finde egenfunktioner er, at vi får os et redskab vi måske kan bruge til at analysere systermer.
            Egenfunktioner, er funktioner som vi kan beskrive ved multiplikation istedet for konvolution.
            Jeg tager $x[n] = z^n$ som vores funktion.
            Jeg bruger den kommukative egenskab at 
            \[y[n] = x\star h = h\star x\]
            \[\convolutionSym{2}{-\infty, \infty}{h, x}\]
            \[x[n-k] = z^{n-k}\]
            \[y[n] = \infsum{k}{h[k]*z^{n-k}}\]
            \[y[n] = \infsum{k}{h[k]*z^{-k}*z^n}\]
            Som leder os til at beskrive en z transformation
            \[H(z) = h[k]*z^{-k}\]
            \[y[n] = \infsum{k}{H(z)*z^n}\]
            Og det var inputtet vi startede med. 
            \[y[n] = \infsum{k}{H(z)*x[n]}\]
        \end{underrubrik}
    \end{Udklip}
    \begin{underrubrik}[Z transformation - Kausulitet]
        Givet et z transformeret input: 
        \[X(z) = \frac{1 + z^-1}{(1 - z^-1)*(1 - 0,5z*-1)}\]
        Så ved vi at det kan beskrives som to simple funktioner, med linearitet til hver at have en amplitude koefficient på sig. 
        I så fald så ved vi, at hvis $|z| > |a|$, så er inputtet en kausul serie. Antikausult hvis $|z| < |a|$, med a her værende 1. 
        Og så kan vi konkludere transformationen.
        \[x[n] = 4u[n] - 3(\frac{1}{2})^n*u[n]\]
    \end{underrubrik}
    \begin{underrubrik}[Z transformation - Eksponentiel aftagende]
        \[X(z) = \filterZ{7/9}{1, 2} + \filterZ{2/9}{1, -1} \]
        Og jeg skal bestemme det for alle dens mulige ROCs. Jeg ved at det er en eksponentiel aftagende funktion, så lad mig se hvordan den er beskrevet.
        
        \[x_1[n] = a^n*u[n] \transformation{Z} X(z) = \sum_{n=-\infty}^{-\infty}  {a^n*u[n]*z^{-n}} \]
        \[x_1[n] \transformation{Z} X(z) = \sum_{n=0}^{-\infty}                        {a^n*z^{-n}} \]
        \[|\frac{a}{z}| < 1, \tab{0} |z| > |a|                                                         \]
        \[================================                                                          \]
        \[x_1[n] = a^n*u[n] \transformation{Z} X_1(z) = \frac{1}{1 - a*z^{-1}}, \tab{0} |z| > |a|      \]
        \[================================                                                          \]
    \end{underrubrik}
    \begin{underrubrik}[Z transformation - Anti kausul eksponentiel aftagende]
        \[x_2[n] = a^{-n}*u[-1-n] \transformation{Z} X_2(z) = \sum_{n=-\infty}^{\infty} {a^{-n}*u[-1-n]*z^{-n}}                     \]
        \[x_2[n] \transformation{Z} X_2(z) = \sum_{n=-\infty}^{-1}                      {a^{-n}*z^{-n}}                             \]
        \[X_2(z) = \sum_{m = 1}^{\infty}                                                {a^{m}*z^{m}}                               \]
        \[X_2(z) = \sum_{m = 0}^{\infty}                                                {(a*z)^m}                                - 1\]
        \[|a*z| < 1, \tab{0} |z| < |\frac{1}{a}|                                                                                       \]
        \[X_2(z) = \frac{1}{1 - a*z} - 1                                                                                            \]
        \[X_2(z) = \frac{1}{1 - a*z} - \frac{1 - a*z}{1 - a*z}                                                                      \]
        \[X_2(z) = \frac{a*z}{1 - a*z}                                                                                              \]
        \[X_2(z) = \frac{z}{\frac{1}{a} - z}                                                                                        \]
        \[X_2(z) = \frac{1}{\frac{1}{a}*z^{-1} - 1}                                                                                 \]
        \[X_2(z) = -\frac{1}{1 - \frac{1}{a}*z^{-1}}                                                                                \]
        \[=====================================                                                                                     \]
        \[x_2[n] = a^{-n}*u[-1-n] \transformation{Z} X_2(z) = -\frac{1}{1 - \frac{1}{a}*z^{-1}}, \tab{0} |z| < |\frac{1}{a}|           \]
        \[=====================================                                                                                     \]

    \end{underrubrik}
    \begin{underrubrik}[Z transformation - Anti kausul eksponentiel]
        \[x_3[n] = a^n*u[-1-n] \transformation{Z} X_3(z) = \sum_{n=-\infty}^{\infty}     {a^n*u[-1-n]*z^{-n}}\]
        \[x_3[n] \transformation{Z} X_3(z) = \sum_{n=-\infty}^{-1}                               {a^n*z^{-n}}\]
        \[X_3(z) = \sum_{m = 1}^{-\infty}                                                   {(\frac{z}{a})^m}\]
        \[X_3(z) = \sum_{m = 0}^{-\infty}                                               {(\frac{z}{a})^m} - 1\]
        \[|\frac{z}{a}|<1, \tab{0} |z|<|a|\]
        \[X_3(z) = \frac{1}{1 - \frac{z}{a}} - 1\]
        \[X_3(z) = \frac{1}{1 - \frac{z}{a}} - \frac{1 - \frac{z}{a}}{1 - \frac{z}{a}}\]
        \[X_3(z) = \frac{\frac{z}{a}}{1 - \frac{z}{a}}\]
        \[X_3(z) = \frac{z}{a - z}\]
        \[X_3(z) = \frac{1}{a*z^{-1} - 1}\]
        \[X_3(z) = - \frac{1}{1 - a*z^{-1}}\]
        \[====================================\]
        \[x_3[n] = a^n*u[-1-n] \transformation{Z} X_3(z) = - \frac{1}{1 - a*z^{-1}}, \tab{0} |z|<|a|\]
        \[====================================\]
    \end{underrubrik}
    \begin{underrubrik}[Z evaluering - Egenfunktion brugt til at evaluere harmonisk signal på simpel vis] 
        \[x[n] = 2cos(0.52n + 60°) = 2 \frac{e^{j(0.52 + 60°)} + e^{j(0.52 + 60°)}}{2}\]
        \[x[n] = e^{j(0.52n + 60°)} + e^{-j(0.52n + 60°)}\]
        \[x[n] = e^{j0.52n}e^{j60°} + e^{-j0.52n}e^{-60j°}\]
        \[x[n] = (e^{j0.52})^ne^{j60°} + (e^{-j0.52})^ne^{-60j°}\]
        Eigenfunktioner er hvor, at outputtet bliver til et forstærket signal af sig selv ud fra $H(z)$. Med den tankegang så vil et signal med faseskift kunne skrives som. 
        \[y[n] = H_1(z_0)z_0^ne^\theta\] 
        \[y[n] = H_2(z_0)z_0^n\]
        Hvor $H_2(z_0) = H_1(z_0)e^j\theta$, så på den måde, så vil en fase bare kunne beskrives som et nyt systems påvirkning.\\
        Set i den kontekst, så følger et harmonisk signal med fase stadigvæk formen: 
        \[x[n] = z_0^n\]
        Med det prøvet bevist så har jeg nu. 
        \[y[n] = H_1(z_0)z_0^ne^\theta\] 
        \[H(z) = \frac{0.19}{1 + 0.81z^{-2}}\]
        \[y[n] = \frac{0.19}{1 + 0.81(e^{j0.52})^{-2}}(e^{j0.52})^ne^{1/3j\pi} + \frac{0.19}{1 + 0.81(e^{-j0.52})^{-2}}(e^{-j0.52})^ne^{-1/3j\pi}\] 
        \[y[n] = \frac{0.19*e^{1/3j\pi}}{1 + 0.81e^{-2j0.52}}(e^{j0.52})^n + \frac{0.19*e^{-1/3j\pi}}{1 + 0.81e^{2j0.52}}(e^{-j0.52})^n\] 
        Med wolfram og python har jeg fundet koefficienterne: 
        \[y[n] = (0.308 + 0.153j) * e^{0.52jn} + (0.038 - 0.0188j) * e^{-0.52jn}\]
        \color{red} Jeg ved, at jeg har glemt at konvertere 0.52 til radianer, som jeg gjorde med vinklen, men idéen er den samme.
        
    \end{underrubrik}
    \begin{underrubrik}[DTFT - \text{\boldmath{$n*(a)^nu[n]$}}]
        Den her kan løses på flere forskellige måder. 
        \[Metode 1\]
        \[-----------------------------------\]
        \[X(e^{j\omega}) = \sum_{n = -\infty}^{\infty}x[n]e^{-j\omega*n}\]
        \[X(e^{j\omega}) = \sum_{n = -\infty}^{\infty}n*(a)^nu[n]e^{-j\omega*n}\]
        \[X(e^{j\omega}) = \sum_{n = 0}^{\infty}n*(a)^ne^{-j\omega*n}\]
        \[X(e^{j\omega}) = \sum_{n = 0}^{\infty}n*(ae^{-j\omega})^n\]
        En geometrisk serie gælder for koefficienten ikke lige med 1. 
        \[\sum_{k=0}^{\infty}k a^{k}=\frac{a}{(1-a)^{2}}\]
        Den kompleks eksponentielle funktion har altid længden 1 så det er kun afhængigt af a i det her tilfælde.
        \[a \neq 1,\tab{0} X(e^{j\omega}) = \sum_{n = 0}^{\infty}n*(ae^{-j\omega})^n = \frac{ae^{-j\omega}}{(1 - ae^{-j\omega})^2}\]
        \[==============\]
        \[X(e^{j\omega}) = \frac{ae^{-j\omega}}{(1 - ae^{-j\omega})^2}\]
        \[==============\]\\\\

        \[Metode 2\]
        \[-----------------------------------\]
        Det kan også ses som differentation af frekvens. Jeg har at 
        \[n x[n] \transformation{F} j{\frac{d X(e^{j\omega})}{d\omega}} \]
        \[f(\omega) = 1/\omega\]
        \[g(\omega) = 1 - ae^{-j\omega}\]
        Så har jeg jo egentligt at: 
        \[n x[n] \transformation{F} \frac{d(f(g(\omega)))}{d\omega}\]
        Og det er differentationen af en sammensat funktion og er givet som.
        \[\frac{df}{d\omega}(g(x)) * \frac{dg}{d\omega}\]
        \[\frac{df}{d\omega}(\omega) = -1/(\omega^2)\]
        \[\frac{dg}{d\omega}(\omega) = - (-j)ae^{-j\omega}\]
        \[\frac{dg}{d\omega}(\omega) = jae^{-j\omega}\]
        \[X(e^{j\omega}) = j * (-1)/((1 - ae^{-j\omega})^2) * jae^{-j\omega}\]
        \[X(e^{j\omega}) = (-1) * (-1 * ae^{-j\omega})/((1 - ae^{-j\omega})^2)\]
        \[=============\]
        \[X(e^{j\omega}) = \frac{ae^{-j\omega}}{(1 - ae^{-j\omega})^2}\]
        \[=============\]
        Et ekstra trik er en udvidelse af sammenhængen funktioner. 
        \[H(x)  = \frac{1}{f(x)}\]
        \[H(x)' = \frac{f(x)'}{f(x)^2}\]
        Og det havde været en hurtig måde at se det på. 
        \[f(\omega) = 1 - ae^{-j\omega}\]
        \[H(\omega)' = \frac{-jae^{-j\omega}}{(1 - ae^{-j\omega})^2}\]
        \[j\frac{H'}{d\omega} = j*(-j)*\frac{ae^{-j\omega}}{(1 - ae^{-j\omega})^2}\]
        \[============\]
        \[j\frac{H'}{d\omega} = \frac{ae^{-j\omega}}{(1 - ae^{-j\omega})^2}\]
        \[============\]

        Så det var 3 måder at løse for den på. 
    \end{underrubrik}\clearpage
    \begin{underrubrik}[Filter strukturer - Direct Form II i z domæne]
        \color{teal} Lavet ud fra opgave 9.29, hvor jeg skulle beskrive samlet system ud fra dets blocke. \color{black}
        \figseksoghalvfjerds{0.4}
        System b skal findes, og jeg det er en blanding mellem DF I systemet og DF II systemet som Second order system. Derfor skal jeg selv udlede den. 
        Midter blocken er et DF II system.
        I bogen har de udledt det for et DF II system: 
        \[Y(z) = H_1(z)H_2(z)X(z) = H_1(z)W(z)\]
        \[W(z) = H_2(z)X(z) = \frac{1}{1 - \sum_{k = 1}^{N}{a_kz^{-k}}}\]
        \[Y(z) = \sum_{k = 0}^{M}{b_kz^{-k}W(z)}\]
        W(z) er konstant i sidste sum og da summer af brøkker af samme nævner kan sættes sammen til en brøk, så får jeg transformationen til at være: 
        \[=============\]
        \[Y(z) = \frac{\sum_{k = 0}^{M}{b_kz^{-k}}}{1 - \sum_{k = 1}^{N}{a_kz^{-k}}}\]
        \[=============\]
    \end{underrubrik}
    
\end{Udledninger}