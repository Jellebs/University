\begin{rubrik}[Eksamenssæt 2021]
    Strategi:
        Signaler og systemer egenskaber skal beskrives hvis de bruges.
        Ellers så skal der bare regnes igennem
    \begin{eksamensOpgave}[Antag, at et LTI system har impulsresponset]
        \[h[n] = [4, -5], \tab{0} n = 0, 1\]
        Der sendes et signal 
        \[x[n] = [1, 3, 2], \tab{0} n = 0, 1, 2\]
        igennem LTI systemet. Som sædvanligt angiver pilen n = 0.
        \begin{UnderOpgave}[Beregn \text{$y[n] = x[n] \star h[n]$} med "papir og blyant, hvis mellemregninger]
            \[y[n] = x[n] \star h[n] = \sum_{k = - \infty}^{\infty}{x[k]h[n - k]}\]
            \[\begin{vmatrix}% {c|c|ccccc}
                n        & -2 & -1 & 0   & 1   & 2   & 3 \\
                h[n]     &  0 &  0 & 4   & 5   & 0   & 0 \\
                x[n]     &  0 &  0 & 1   & 3   & 2   & 0 \\
                \hline
                h[k - 0] &  0 &  5 & 4   & 0   & 0   & 0 \\
                h[k - 1] &  0 &  0 & 5   & 4   & 0   & 0 \\
                h[k - 2] &  0 &  0 & 0   & 5   & 4   & 0 \\
                h[k - 3] &  0 &  0 & 0   & 0   & 5   & 4 \\
                h[k - 4] &  0 &  0 & 0   & 0   & 0   & 5 \\
                \hline
                y[0]     &  0 &  0 & 4*1 & 0   & 0   & 0 \\
                y[1]     &  0 &  0 & 5*1 & 3*4 & 0   & 0 \\  
                y[2]     &  0 &  0 & 0   & 5*3 & 4*2 & 0 \\
                y[3]     &  0 &  0 & 0   & 0   & 5*2 & 0 \\
                y[4]     &  0 &  0 & 0   & 0   & 0   & 0 \\
            \end{vmatrix}\]
            \[y[n] = [4, 17, 23, 10], \tab{0} n = 0, ... 3\]
            Det kunne jeg have løst noget hurtigere, men der er den. 
        \end{UnderOpgave}
        \begin{UnderOpgave}[Et andet signal er givet ved \text{$x_2[n] = 3x[n]$}]
            Hvordan er $h[n] \star x[n]$ og $h[n] \star x_2[n]$ relateret? 
            \[y[0] = h[0] * x_2[0] = 3*4\]
            \[y[1] = x_2[0] * h[1] + x_2[1]*h[0] = 3 * 5 + 9 * 4\]
            \[y[2] = x_2[1] * h[1] + x_2[2] * h[0] = 9 * 5 + 6 * 4\]
            \[y[3] = x_2[2] * h[1] = 6 * 5\]
            \[y[n] = [12, 51, 69, 30]\]
            Så det er en simpel.
            \[y_2 = 3 * y_1\]
            Og det giver jo også mening, fordi det er en LTI system og det er linearitets princippet.            
        \end{UnderOpgave}
    \end{eksamensOpgave}
    \begin{eksamensOpgave}[Antag, at et filter har systemfunktionen]
        \[H(z) = \filterZto{3, 1}{1, 0.5}\]
        \begin{UnderOpgave}[Bestem differensligningen for filteret]
            \[Y(z)/X(z) = H(z)\]
            \[Y(z) = \filterZto{3, 1}{1, 0.5} X(z)\]
            \[Y(z)(1 + 0.5z^{-1}) = X(z)(3 + z^{-1})\]
            \[y[n] + 0.5y[n - 1] = 3x[n] + x[n - 1]\]
            \[y[n] = 3x[n] + x[n - 1] - 0.5y[n - 1]\]
        \end{UnderOpgave}
        \begin{UnderOpgave}[Sketch signal flow graphen]
            I mit sketch dokument har jeg tegnet det. 
        \end{UnderOpgave}
        \begin{UnderOpgave}[Sketch frekvens karakteristikken]
            Hvis jeg skal beskrive filter strukturen så er den af form: 
            Direkt form I
            \figsyvogfirs{0.23}\\
            Den virker som et høj pass filter.
        \end{UnderOpgave}
    \end{eksamensOpgave}
    \begin{eksamensOpgave}[Antag, at der er givet et talk-through system med en ideel ADC og en ideel DAC som vist nedenfor. Begge har en samplefrekvens på 1000 Hz (Jeg mangler opgave 2)]
        \[x_c(t) -> ADC -> x[n] -> DAC -> y_r(t)\]
        Antag, at input signalet, xc(t) består af tre rentoner, dvs. sinusformede signaler med frekvenser på henholdsvis 300 Hz, 400 Hz og 600 Hz. Alle tre rentoner har samme amplitude.
        \begin{UnderOpgave}[Redegør for indholdet af det rekonstruerede signal \text{$y_r(t)$}]
            Grundet at diskrete filtres spektrum gentager sig selv, så er der en risiko for aliasing. 
            Aliasing vil sige, at dele af det gentagende spektrum overlapper med det første spektrum. Det gælder så for alle steder hvor spektrene krydser. 
            For at undgå aliasing, så skal der gælde at: 
            \[2*f_N < f_s\]
            Hvor fN er det største frekvens element. 
            For det her system så gælder det ikke, da den ene ren tone vil bidrage til støj.
        \end{UnderOpgave}
    \end{eksamensOpgave}
    \begin{eksamensOpgave}[Lad et signal være givet ved]
        \[x[n] = [1, -1, 4, 7], \tab{0} n = 0, ..., 3\]
        \begin{UnderOpgave}[Beregn X(z) og redegør for konvergensen af X(z)]
            Den laver lidt en halv sinc funktion. Men jeg er ikke klog nok til at vide dens opsætning bare sådan. 
            Jeg kunne beregne den analytisk med DFT beregningen med W matricen. Men det er jeg ikke blevet spurgt om. Og dog, ...
            Den måde de først vil have X(z) for så at beregne \[X(e^{j\omega})\]
            \[X[k] = \sum_{n = 0}^{N - 1}{x[n] W_N^{kn}},  k = 0, 1, ..., N - 1\]
            Så hvis jeg laver en matrice hvor 
            \[W = e^{-(j * 2\pi/N) * kn } \in \mathbb{R}^{N x N}\]
            Hvis jeg laver en vander matrice ned af n'erne. 
            \[W_N^n = e^{-(j*2\pi/N) * n} \in \mathbb{N x 1}\]
            Hvis jeg så laver vander på den igen ad kolonnerne får jeg
            \[W_N^{kn} = e^{-(j * 2\pi/N) * kn } \in \mathbb{R}^{N x N}\]
            Så sådan sætter jeg den op. 
            \[
            x[n] = [\begin{array}{cccc}
                1 & -1 & 4 & 7
            \end{array}]
            \]
            \[W_N^{kn} = e^{-(j*2\pi/4)*kn}\]
            \[
            W_N^{kn} = 
            \begin{vmatrix}
                1 &         1 &         1 &       1 \\
                1 & W^{1 * 1} & W^{1 * 2} & W^{1*3} \\
                1 & W^{2 * 1} & W^{2 * 2} & W^{2*3} \\
                1 & W^{3 * 1} & W^{3 * 2} & W^{3*3} \\
            \end{vmatrix}\]
            Og så kan man se sammenhængere med dens gentagelser, men det vil jeg ikke kommentere så meget på.
            \[X = x[n].W_N^{kn}\]
            Ved beregning af søjle produktet med matricen får jeg
            \[X[k] = [\begin{array}{cccc} 11 & -3 + 8j & -1. & -3 -8j\end{array}]\]
            Så hvis det er sådan, at man ville have X(z), så har jeg da vidst lidt DFT
            Beregn $X(e^{j\omega})$. Er det en periodisk eller ikke-periodisk funktion?
            Her bruger jeg så FFT og zero padder den for at få bedre resolution.
            \figotteogfirs{0.4}\\
            Så hvad jeg så i min DFT er jo så hvad jeg ser i mit plot. Den ændring der så lige er, er at jeg beregnede DFTen i intervallet $[0:2\pi]$ og jeg plottede for $[-\pi:\pi]$
            Rykker jeg rundt på X 
            \[X[k] = [\begin{array}{cccc} -1, -3 + 8j, 11 & -3 - 8j \end{array}]\]
            Hvor den tager frekvenserne 
            \[\omega_k = [-\pi, 0.5\pi, 0, 0.5\pi]\]
            Så kan jeg jo se, at det er det jeg har fået i mit plot. \\\\
            Så til om funktionen er periodisk. Ja det er den jo, men det er på grund af at den er samplet. Spektret vil gentage sig i $2\pi$
        \end{UnderOpgave}
    \end{eksamensOpgave}
    \begin{eksamensOpgave}[Sampling og aliasing]
        Antag, at der et givet et signal med en varighed på 1 sekund. Signalet er samplet med 1000Hz. 
        Det vides, at signalet indeholder en rentone, dvs, et sinusformet signal, med en frekvens
        på F0 = 100 Hz. Antag, at signalet også indeholder en anden rentone med en frekvens på
        $F0 + \Delta F$. 
        \begin{UnderOpgave}[Hvad er den mindste værdi for $\Delta F$ , hvor der kan ses to toppe i frekvensspektret af signalet?]
            Jeg tror ikke rigtigt jeg forstår spørgsmålet. Havde det været noget med interpreterings funktionen, 
            så vil det betyde, at der være firkanter om impulserne i frekvens spektret. 
            I det tilfælde, så vil interpreterings funktionen laves, så dens første zero crossing falder ved Ts, \\
            Så vil firkanterne være $2\pi/T_0$ bredde. De eneste steder som den ikke må betræde er så: 
            \[[\omega_0 - \pi/T_0: \omega_0 + \pi/T_0], [-\omega_0 -\pi/T_0 : -\omega_0 + \pi/T_0]\]\\
            \[H_R(\omega) = rect[(\omega - \omega_0)/(2\pi/T_0)] + rect[(\omega + \omega_0)/(2\pi/T_0)]\]
            \[H_R(\omega) = rect[(\omega - \omega_0)/(\omega_0)] + rect[(\omega + \omega_0)/(\omega_0)]\]
            Med den tankegang, så skal der være to halve bredder mellem den og den anden ren tone. 
            \[\Delta w_{min} = 2\pi/T_0\]
            \[\Delta f_{min} = 1/T_0 = 200Hz\]
            Så hvis signalet skal rekonstrueres og at alle værdierne toner skal kunne ses klart i frekvenspektret, så skal 
            \[=======\]
            \[\Delta f = 200Hz \]
            \[=======\]
        \end{UnderOpgave}
    \end{eksamensOpgave}
\end{rubrik}\setcounter{eksamensOpgave}{0}

\begin{rubrik}[Eksamenssæt2021 - Rettet]
    \begin{eksamensOpgave}
        
    \end{eksamensOpgave}
    \begin{eksamensOpgave}
        
    \end{eksamensOpgave}
    \begin{eksamensOpgave}
        
    \end{eksamensOpgave}
    \clearpage
    \begin{eksamensOpgave}[Lad et signal være givet ved]
        \[x[n] = [1, -1, 4, 7], \tab{0} n = 0, ...0, 3\]
        \begin{UnderOpgave}[Beregn X(z) og redegør for konvergensen af X(z)]
            \color{teal}Se her troede jeg, at $z = e^{j\omega}$ så jeg forstod ikke rigtig, at der var blevet sat op for begge.
            Når chatten nu fortæller det, så kan jeg godt huske, at det ikke var sådan at z transformationen var sat op. 
            For vi snakker om ROC for $|z| > a$ eller $|z| < a$. Det er så sådan, at DFT er tilfældet hvor, $|z| = 1, z = e^{j\omega}$
            Så jeg løste egentlig opgave b to gange. \\
            Nej jeg skal tilbage til opbygningen af z transformationen\color{black}
            \[X(z)=\sum_{n=-\infty}^{\infty}x[n]z^{-n}\]
            \[X(z) = 1 - z^{-1} + 4z^{-2} + 7z{-3}\]
            For denne z transformation er der kun nulpunkter og der er der stabilitet for alle z'er.\\
            \[X(z) = 1 - z^{-1} + 4z^{-2} + 7z{-3}, \tab{0} \text{ROC: alle z}\]
        \end{UnderOpgave} 
        \begin{UnderOpgave}[Beregn \text{$X(e^{j\omega})$}. Er den periodisk eller ikke-periodisk funktion?]            
            Jeg substituere for $z = e^{j\omega}$ 
            \figotteogfirs{0.23}\\\\
            Selvfølgelig for en diskrete funktion vil spektret gentage sig i $2\pi$. Ellers så er spektret ikke mere gentagende end det.
        \end{UnderOpgave}
        \begin{UnderOpgave}[Et andet signal \text{$x_2[n]$} har Fourier-transformationen]
            Udtryk 4x2[n - 1] + 3x2[3 - n] i forhold til dens fourier transformation. \\
            Det her er en typisk system karakteristisk opgave. Jeg har ikke signalet men skal beskrive den så godt jeg kan alligevel.
            Jeg antager systemet er LTI. Der er tidsforskydelse og tidsinvertering.
            Jeg skal have gang i folding egenskaben også. 
            \[x[n] \transformation{Z} X(z)\]
            \[x1[n] = x[-n] \transformation{Z} X(1/z)\]
            \[x2[n] = x1[n - 3] = x[-(n - 3)] = x[3 - n] \transformation{Z} z^{-3}X(1/z)\]
            \[x3[n] = 2 * x2[n] = 2 * x1[n - 3] = 2 * x[-(n - 3)] = 2 * x[3 - n] \transformation{Z} 2 * z^{-3}X(1/z)\]
            Det var den sværeste, den anden er relativ simpel med linearitet og tidsforskydelse.
            \[x_2[-n] \transformation{Z} X(1/z)\]
            \[Y_2(z) = X_2(z) * 4 z^{-1} + 2 * z^{-3}X_2(1/z)\]
            \[Y_2(e^{j\omega}) = X_2(e^{j\omega}) * 4 e^{-j\omega} + 2 * e^{-3 j\omega}X_2(e^{-j\omega})\]
            Med forskellige antagelser af signalet, så kan man reducere udtrykket.
            F.eks ved at kræve at signalet er reelt, så har jeg at: 
            \[X_2(e^{j\omega}) = X_2^\star(e^{-j\omega})\]
            Det står der ikke explicit. 
            Hvis jeg også kræver at signalet er symmetrisk, så den nu er reel og symmetrisk. Jeg er ikke helt sikker på hvad det vil kræve, men så: 
            \[X_2(e^{j\omega}) = X_2(e^{-j\omega})\]
            Så
            \[Y_2(e^{j\omega}) = X_2(e^{j\omega}) * 4 e^{-j\omega} + 2 * e^{-3 j\omega}X_2(e^{j\omega})\]
            \[Y_2(e^{j\omega}) = X_2(e^{j\omega}) *( 4 e^{-j\omega} + 2 * e^{-3 j\omega})\]
            Og så kan jeg manipulere signalet.
            \[Y_2(e^{j\omega}) = e^{-2j\omega} X_2(e^{j\omega}) * ( 4 e^{j\omega} + 2 * e^{- j\omega})\]
            \[Y_2(e^{j\omega}) = e^{-2j\omega} X_2(e^{j\omega}) * ( 4 e^{j\omega} + 2 * e^{- j\omega})\]
            Havde det ikke været fordi, at det var to forskellige amplituder, så havde jeg kunnet beskrive den som en cos funktion. 
        \end{UnderOpgave}
    \end{eksamensOpgave}
    \begin{eksamensOpgave}[Frekvensspektre, resolution og flere frekvensdele]
        Antag, at der et givet et signal med en varighed på 1 sekund. Signalet er samplet med 1000Hz. Det vides, at signalet indeholder en rentone, dvs, et sinusformet signal, med en frekvens
        på F0 = 100 Hz. Antag, at signalet også indeholder en anden rentone med en frekvens på $F0 + \Delta F$.
        Jeg misforstod den her opgave. Det handlede ikke om sampling, det handler om resolution i frekvensspektret.
        \begin{UnderOpgave}[Hvad er den mindste værdi \text{$\Delta F$} hvor begge toppe kan ses]
            \[N = 1s/(1/1000Hz) = 1000\]
            \[\Delta \omega = 2\pi/N\]
            Man kan argumentere for, at man måske vil kunne skelne mellem de to, så der skulle være en frekvens imellem dem...
            Men altså hvis man kender bredden som jeg lige har fundet, så kan man se at det er to værdier. 
            \[=========\]
            \[\Delta f_0 = 1/1000\]
            \[=========\]
        \end{UnderOpgave}
        \begin{UnderOpgave}[Hvordan ændrer svaret på ovenstående sig, hvis signalets længde øget?]
            Hvis den øges og ikke formindskes, så vil resolutionen blive bedre og bedre, og så behøves der ikke at være lige så stor afstand som havde man ikke øget den. 
            Hvis man mindsker den, så vil det have modsat effekt. 
            \[T \uparrow \tab{0} \Rightarrow \tab{0} N \uparrow \tab{0} \Rightarrow \tab{0} f_0\downarrow\]           
        \end{UnderOpgave}
        Antag at der er mulighed for at anvende en vinduesfunktion, f.eks. et Hann vindue, i forbindelse med frekvensanalysen
        \begin{UnderOpgave}[Hvilken indvirkning har det på svarene på ovenstående to spørgsmål, hvis der ganges en vinduesfunktion på signalet inden frekvensanalysen?]
            Vinduer undertoner sidelopsne, men på bekostning af at frekvenspektret flades ud. 
            Så i forhold til de spørgsmål jeg blev stillet før, så vil en vindue mindske resolutionen. Resolutionen skal så øges det mere pga. den også.
        \end{UnderOpgave}
        
    \end{eksamensOpgave}
\end{rubrik}\setcounter{eksamensOpgave}{0}

\begin{rubrik}[Eksamenssæt2022]
    \begin{eksamensOpgave}[Convolution i LTI system]
        \[h[n] = [1, 4, -3], \tab{0} n = 0, 1, 2\]
        Der sendes et signal igennem
        \[x[n] = [2, 3], \tab{0} n = 0, 1\]
        \begin{UnderOpgave}[Beregn y0 og y1]

            \[y[n]=\sum_{k =-\infty}^{\infty}x[k]h[n-k]\]
            \[==============================\]
            \[y[0] = \sum_{k =-\infty}^{\infty}x[k]h[0-k] = x[0]h[0] = 2\]
            \[y[1] = \sum_{k =-\infty}^{\infty}x[k]h[1-k] = x[0]h[1] + x[1]h[0] = 8 + 3\]
            \[==============================\]            
        \end{UnderOpgave}
        Et andet signal er givet ved
        \[x_2[n] = [0, 2, 3]\]
        \begin{UnderOpgave}[Hvordan er \text{$h[n]\star x[n]$} og \text{$h[n] \star x_2[n]$} relateret?]
            Så altså 
            \[x_2[n] = x[n - 1]\]
            Og for LTI systemer hvor der er time invariance medføre forskydningen i tiden i inputtet, den samme forskydning i tiden i outputtet
            \[y_2[n] = y[n - 1]\]
            
            
        \end{UnderOpgave}

    \end{eksamensOpgave}
    \begin{eksamensOpgave}[Signal flow graf]
        \begin{UnderOpgave}[Opstil differensligningen og bestem z transformationen af h]
            Systemet er et DF 1 system. 
            \[y[n] = x[n] + (-2x[n - 1]) + (-0.5y[n - 1])\]
            \[y[n] = x[n] - 2x[n - 1] - 0.5y[n - 1]\]
            \[y[n] + 0.5y[n - 1]= x[n] - 2x[n - 1]\]
            \[Y(z) * (1 + 0.5z^{-1}) = X(z) * (1 - 2z^{-1})\]
            \[H(z) = Y(z)/X(z) = \frac{1 - 2z^{-1}}{1 + 0.5z^{-1}}\]            
        \end{UnderOpgave}
        \begin{UnderOpgave}[Redegør for antallet af poler og nulpunkter i filteret]
            Filteret har 1 pol og 1 nulpunkt.             
        \end{UnderOpgave}
        Antag at der sendes et signal $x[n] = cos(\frac{\pi}{4} n)$ igennem filteret. 
        \begin{UnderOpgave}[Redegør for, om amplituden af signalet vil være dæmpet, uændret eller forstærket efter filteret]
            \figniogfirs{0.3}
            Viser frekvensresponsen til frekvenser i intervallet $[-\pi: \pi]$
            For et signal med $\omega_0 = \pi/4$, så er signalet omkring uændret, en lille smule forstærket.
        \end{UnderOpgave}
    \end{eksamensOpgave}
    \begin{eksamensOpgave}[System karakteristikker af to systemer]
        \[y_1[n] = x^2[n]\]
        \[y_2[n] = x[n]cos(\pi/3 n)\]
        \begin{UnderOpgave}[Redegør for om systemerne er LTI systemer]
            I første system ser jeg et problem i linearitet. 
            For at den gælder så skal en faktor på inputtet være det samme som en faktor på outputtet.
            \[y_1[n] = x_1^2[n]\]
            Ændring i input
            \[x_2[n] = ax_1[n]\] 
            \[y_2[n] = x_2^2[n]\]
            Ændring i output
            \[y_3[n] = ay_1[n]\]
            Testen, kan jeg sige at
            \[y_2[n] = y_3[n]\] 
            \[x_2^2[n] = ay_1[n]\] 
            \[(ax_1[n])^2 = ax_1^2[n]\] 
            \[a^2x_1^2[n] \neq ax_1^2[n]\] 
            Så derfor kan jeg bekræfte, at første system ikke er et LTI system.
            Så til det andet system.
            Her ser jeg en tydelig mangle i Time invariancen.
            \[y_2[n] = x_1[n]cos(\pi/3 n)\]
            Ændring i inputtet
            \[x_2[n] = x_1[n - n_0]\]
            \[y_3[n] = x_2[n]cos(\pi/3 n)\]
            \[y_3[n] = x_1[n - n_0]cos(\pi/3 n)\]
            Ændring i outputtet
            \[y_4[n] = y_2[n - n_0]\]
            \[y_4[n] = x_1[n - n_0]cos(\pi/3 (n - n_0))\]
            Testen: 
            \[y_3[n] = y_4[n]\]
            \[x_1[n - n_0]cos(\pi/3 n) \neq x_1[n - n_0]cos(\pi/3 (n - n_0))\]
            Så ja, der er fejl i Tids Invariancen\\
            ===================\\
            Begge systemer fejler LTI kravene\\
            ===================\\
        \end{UnderOpgave}
        Et tredje system har differensligningen $y_3[n] = 0.5y[n - 1] + 2x[n] + x[n - 1]$.
        \begin{UnderOpgave}[Redegør for om systemet er et lineær fase filter]
            Filteret er et IIR filter, og det har nogle krav omkring lineær fase, som gør den ustabil. 
            Derfor kan den ikke være et lineær fase filter         
        \end{UnderOpgave}
    \end{eksamensOpgave}
    \begin{eksamensOpgave}[ADC (mangler sidste opgave)]
        \[F_s = 1000Hz\]
        \[x_c(t) = cos(2\pi * 600Hz * t)\]
        \begin{UnderOpgave}[Redegør for at det samplede signal kan beskrives ved]
            \[x[n] = cos(2\pi \frac{2}{5} n)\]
            Min udledning: 
            \[x[n] = x_c(n * T_s) = cos(2\pi * 600Hz * n * T_s)\]
            \[x[n] = cos(2\pi * 600Hz * n * 1/1000)\]
            \[===========\]
            \[x[n] = cos(2\pi * \frac{6}{10} n )\]
            \[===========\]
            Hmm... Jeg tror der er en fejl i frekvensen. For at det skulle være rigtig var der brug for at det analoge signal havde en $F_0 = 400Hz$.
        \end{UnderOpgave}
        \begin{UnderOpgave}[Argumenter for peridiciteten]
            \[y[n] = x[n] - x[n - N] = 0\]
            Jeg skal bestemme den for den mindste N. 
            \[cos(2\pi \frac{6}{10} (n + N))\]
            \[cos(\frac{6}{5}\pi (n + N))\] 
            Jeg bruger eulers relation
            \[cos(\frac{6}{5}\pi (n + N)) = \frac{e^{j * \frac{6}{5}\pi (n + N)} + e^{- j * \frac{6}{5}\pi (n + N)}}{2}\]
            \[0.5 * e^{j * \frac{6}{5}\pi (n + N)} + 0.5 e^{- j * \frac{6}{5}\pi (n + N)}\]
            \[0.5 * e^{j * \frac{6}{5}\pi N } * e^{j * \frac{6}{5}\pi * n} + 0.5 * e^{-j * \frac{6}{5}\pi N } * e^{-j * \frac{6}{5}\pi * n}\]
            Jeg har bare brug for, at frekvensen giver 1. I det tilfælde, så er de negative frekvenser lige med de positive frekvenser... pga. cos... 
            \[e^{j * \frac{6}{5}\pi N }\] 
            \[1.2 * N \mod 2 = 0\]
            Så er den et lige antal pier omkring.
            Det sker for N = 5
            \[e^{j * \frac{6 * 5}{5}\pi} = 1\]
            Så ud fra den omvej så kan jeg så beskrive periodiciteten af signalet for N = 5. 
            \[========================\]
            \[y[n] = x[n] - x[n - k * 5] = 0, \tab{1} alle \tab{0} k \]
            \[========================\]
        \end{UnderOpgave}
        Signalet x[n] rekonstrueres nu med en ideel DAC, men med en samplingfrekvens på 2000Hz
        \begin{UnderOpgave}[Argumenter for, at når signalet rekonstrueres får man en ren sinustone og bestem sinustones frekvens i Hz]
            Rekonstruerings funktionen er en sinc funktion, som har nulpunkter i multiplika af kT, hvor $T = 1/F_H$
            Så signalet her som har $F_h = 1/600$
            I frekvens domænet vil den så være en rektangulær funktion. 
            \[rect(\frac{\omega}{\pi * F_H})\]
            Dens bredde er $2 * \pi * F_H$. Det step er hvad der gør, at sampling frekvens bliver nødt til at være dobbelt af det højeste frekvens komponent er. 
            Og generelt egentlig også bare nyquist ratens krav.

            Frekvensen præcist???? 
            
        \end{UnderOpgave}
    \end{eksamensOpgave}
    \begin{eksamensOpgave}[DFT ]
        \begin{UnderOpgave}[Beregn \text{$W_N$} for N = 4 og beregn derefter DFT transformationen af signalet]
            \[x[n] = 3 - n, \tab{0} n = 0, 1, 2, 3\]
            Det er ikke tilstrækkeligt bare at vise resultaterne, vis beregningerne.\\\\
            \[W_N^{kn} = e^{ - j \frac{2\pi}{N} kn}\]
            \[W_N^{kn} = [e^{0}, e^{ - j \frac{2\pi}{4} * 1}, e^{ - j \frac{2\pi}{4} * 2},  e^{ - j \frac{2\pi}{4} * 3}, e^{ - j \frac{2\pi}{4} 2 * 2}, e^{ - j \frac{2\pi}{4} 2 * 3}, e^{ - j \frac{2\pi}{4} 3 * 3}]\]
            \[W_N^{kn} = [1, -j, -1, j, 1, -1, -j]\]
            Er de mulige værdier, i den rækkefølge.
            \[W_N = \begin{vmatrix}
                1 &  1 &  1 &  1 \\
                1 & -j & -1 &  j \\
                1 & -1 &  1 & -1 \\
                1 &  j & -1 & -j 
            \end{vmatrix}\]
            For god ordens skyld så beregner jeg den også med numpys vander matrice. 
            \[W_N^n = [e^{ - j \frac{2\pi}{N} * 0}, e^{ - j \frac{2\pi}{N} * 1}, e^{ - j \frac{2\pi}{N} * 2}, e^{ - j \frac{2\pi}{N} * 3}]\]
            Og så smider jeg den ind i en vander matrice. 
            \[W_N^{kn} = [(W_N^n)^0, (W_N^n)^1, (W_N^n)^2, (W_N^n)^3].T \]
            Min beregning der siger det samme, så jeg må have lavet matricen korrekt. 
            \[==============\]
            \[W_N = \begin{vmatrix}
                1 &  1 &  1 &  1 \\
                1 & -j & -1 &  j \\
                1 & -1 &  1 & -1 \\
                1 &  j & -1 & -j 
            \end{vmatrix}\]
            \[==============\]
            \[x[n] = [3, 2, 1, 0]\]
            \[X[k] = W_N . x[n]^T\]
            Og ja, den undlader jeg lige selv at beregne. Jeg vil kunne komme til at lave unødvendige fejl beregninger i matrix multiplikationen. Det er ikke det det handler om. 
            \[X[k] = np.inner(W_N, x)\]
            \[X[k] = [6, 2 - 2j, 2 , 2 + 2j]\]
            Finder den for mig. 
            Men numpy.fft.fft forsikre jeg mig, at resultatet er rigtig, og det giver det samme. 
            \[===============\]
            \[X[k] = [6, 2 - 2j, 2 , 2 + 2j]\]
            \[===============\]
        \end{UnderOpgave}
        \[Y[k] = W_N . y[n]\]
        \[W_N^{-1}Y[k] = y[n]\]
        \begin{UnderOpgave}[Beregn y og kommenter]
            \[y[n] = [1, 0, 0, 0]\]
            Og når man ser på fourier transformationen af en impuls, så er det klart, at det bare virker som dc i frekvensspekret.\\
            Havde det ikke være for den korte resolution på 4 i Y, og at jeg havde paddet nogle nuller derind, så havde det set anderledes ud. 
            \[=========\]
            \[y[n] = [1, 0, 0, 0]\]
            \[=========\]
        \end{UnderOpgave}
        \begin{UnderOpgave}[Argumenter for mere effektive algoritmer, såsom fft]
            Grundet at nogle af W gentager sig, så er det muligt at reangere matricen, så beregninger bruges flere gange. \\
            Det er ideen bag fft, hvor de reangere inputtet i lige og ulige, og W på samme måde. 
            Det tager antal bereginger fra $O(N^2)$ til $O(n log_2(n))$. \\\\
            En anden algoritme som er god, hvis man ikke har stort behov for alle frekvens dele som er med til at opbygge et signal, men kun de af størst betydning,
            så kan man bruge Goertzel algorithmen, som netop tager sig af det. 
        \end{UnderOpgave}
    \end{eksamensOpgave}
\end{rubrik}\setcounter{eksamensOpgave}{0}

\begin{rubrik}[Eksamenssæt2022 - Forbedringer]
    \begin{eksamensOpgave}
        
    \end{eksamensOpgave}
    \begin{eksamensOpgave}[LTI system i signal flow]
        \begin{UnderOpgave}[Opstil differensligningen]
            Signal flow diagrammet er i opgaven. Jeg får 
            \[H(z) = \filterZto{1, -2}{1, 0.5}\]            
        \end{UnderOpgave}
        \begin{UnderOpgave}[Redegør for antallet af poler og nulpunkter]
            Filteret går i nul for et nulpunkt.
            Filteret går mod uendelig for en pol. 
            Derfor har filteret en pol og et nulpunkt.             
        \end{UnderOpgave}
        \begin{UnderOpgave}[Redegør for amplitude ændringen af signalet. Dæmpet, uændret eller forstærket?]
            Her var det så, at jeg plottede magnituden. I stedet for så kunne jeg have evalueret det. 
            \[H(e^{j\pi/4}) = \frac{1 - 2*e^{-j\pi/4}}{1 + 0.5e^{-j\pi/4}}\]
            Fjerne det komplekse i nævneren med kompleks konjugerings metoden. Kompleks konjugering på en eksponentiel er det samme som at ændre dens fortegn i eksponenten. 
            \[H(e^{j\pi/4}) = \frac{1 - 2*e^{-j\pi/4}}{1 + 0.5e^{-j\pi/4}} * \frac{1 + 0.5e^{j\pi/4}}{1 + 0.5e^{j\pi/4}}\]
            \[H(e^{j\pi/4}) = \frac{1*1 + 1*0.5e^{j\pi/4} - 2*e^{-j\pi/4} - 2*0.5e^{-j\pi/4}e^{j\pi/4}}{1*1 + 0.5e^{-j\pi/4} + 0.5e^{j\pi/4} + 0.5*0.5e^{-j\pi/4}e^{j\pi/4}}\]
            \[H(e^{j\pi/4}) = \frac{1 + 0.5e^{j\pi/4} - 2*e^{-j\pi/4} - 2*0.5*1}{1 + 0.5e^{-j\pi/4} + 0.5e^{j\pi/4} + 0.5*0.5*1}\]
            \[H(e^{j\pi/4}) = \frac{1 + 0.5e^{j\pi/4} - 2*e^{-j\pi/4} - 2*0.5*1}{1 + 0.5e^{-j\pi/4} + 0.5e^{j\pi/4} + 0.5*0.5*1}\]
            På grund af at $sin(-\theta) = -sin(\theta)$ så vil mine eksponentielle udtryk i nævneren med samme størrelse, have de komplekse gå ud med hinanden. 
            For udtrykket med skiftende amplituder i tælleren så får jeg at
            \[0.5jsin(\pi/4) - (-2jsin(\pi/4)) = 2.5jsin(\pi/4)\]
            Og for dens cos udtryk
            \[0.5cos(\pi/4) - 2*cos(\pi/4) = - 1.5cos(\pi/4)\]
            \[H(e^{j\pi/4}) = \frac{1 - 1.5cos(\pi/4) + 2.5jsin(\pi/4)- 2*0.5*1}{1 + 0.5cos(\pi/4) + 0.5cos(\pi/4) + 0.25}\]
            \[H(e^{j\pi/4}) = \frac{2.5jsin(\pi/4) - 1.5cos(\pi/4)}{1.25 + cos(\pi/4)}\] 
            \[cos(\pi/4) = sin(\pi/4) = \sqrt{2}/2\]
            \[H(e^{j\pi/4}) = \frac{\sqrt{2}/2 * (2.5j - 1.5)}{1.25 + \sqrt{2}/2}\] 
            \[H(e^{j\pi/4}) \approx \frac{1.75j - 1.1}{2}\] 
            \[|H(e^{j\pi/4}) = |0.875j - 0.55| = \sqrt{0.875^2 + 0.552^2} \approx 1.03\]
            Så størrelsen på signalet vil være uændret. 
            Det er meget beregning. Det er rigtigt, men det kunne også have været forkert. 
            En anden måde jeg kunne have gjort det på var: 
            \[|H(e^{j\pi/4})|^2 = (|tæller|^2)/(|nævner|^2)\]
            For den frekvens så har jeg at 
            \[cos(\pi/4) = sin(\pi/4) = sqrt{2}/2\]
            \[tæller = 1 - 2 * \sqrt{2}/2 * (1 - j)\]
            \[|tæller|^2 = (1 - \sqrt{2})^2 + \sqrt{2}^2\]
            \[|tæller| = \sqrt{|tæller|^2} = 1.47\]
            \[nævner = 1 + 0.5 * \sqrt{2}/2 * (1 - j)\]
            \[|nævner|^2 = (1 + \sqrt{2}/4)^2 + \sqrt{2}/4\]
            \[|nævner| = \sqrt{|nævner|^2} = 1.4\]
            \[|H(e^{j\pi/4})| = \sqrt{|H(e^{j\pi/4})|^2} = 1.05\]
            Så det var også en måde at evaluere den på. 
        \end{UnderOpgave}
    \end{eksamensOpgave}
    \begin{eksamensOpgave}
        
    \end{eksamensOpgave}
    \begin{eksamensOpgave}[Sampling med aliasing]
        \begin{UnderOpgave}[Indse at det samplede signal kan beskrives med]
            \[x[n] = cos(2\pi \frac{2}{5})\]
            Se her troede jeg, at det var en fejl. 
            \[x[n] = x_c(n*T_s) = x_c(n/1000) = cos(2\pi * 600/1000 * n)\]
            \[x[n] = cos(2\pi*3/5 * n)\]
            Men fordi at nyquist rate ikke er mødt: 
            \[2 * 600 < f_s\]
            Er falsk
            \[2 * 600 > 1000\]
            Så er der aliasing. 
            Jeg kan finde dens sande frekvens ved: 
            \[F = k*F_s - F_0\]
            Hvor formlen er gældende for F som opfylder nyquist. 
            \[F = 1000 - F_0\]
            Og så har jeg nemlig den sande frekvens.
            \[x[n] = x_c(n/1000) = cos(2\pi * 400/1000 * n)\]
            \[x[n] = cos(2\pi \frac{2}{5} * n)\]
        \end{UnderOpgave}
        \begin{UnderOpgave}[Bevis periodicitet]
            Den lavede jeg rigtigt. N = 5.
        \end{UnderOpgave}
        \begin{UnderOpgave}[Argumenter for at signalet rekonstrueres til en ren sinustone og bestem frekvensen i Hz]
            \[\omega_{norm} = 2\pi f/f_s\]
            \[\omega_{norm} = 2\pi 400/1000 = 800/1000\pi\]
            Er sammenhængen når man tager den til det diskrete univers. 
            Men nu tager jeg det tilbage og ændring sampling frekvensen
            \[\omega_{norm} = 2\pi f/f_s\]
            \[f = \omega_{norm}*f_s/(2\pi)\]
            \[f = (400/1000)*f_s\]
            \[f = (400/1000)*2000\]
            \[f = 800Hz\]
            Så derfor vil det rekonstruerede signal have frekvensen 800Hz. 
            
        \end{UnderOpgave}
        
    \end{eksamensOpgave}
    \begin{eksamensOpgave}
        
    \end{eksamensOpgave}
    
\end{rubrik}\setcounter{eksamensOpgave}{0}

\begin{rubrik}[Eksamenssæt2024 Ordinær]
    \begin{eksamensOpgave}
        \begin{UnderOpgave}[Opskriv det analytiske udtryk for at beregne y1 og angiv hvor mange multiplikationer og additioner beregningen kræver]
            Jeg finder min konvolution sum frem fra min signaler og systemer bog
            \[y[n] = \sum_{k = -\infty}^{\infty}{x[k] h[n - k]}\]
            \[y[1] = \sum_{k = -\infty}^{\infty}{x[k] h[1 - k]}\]
            For det her system har jeg. 
            \[y[1] = x[0]h[1] + x[1]h[0] + x[2]h[-1]\]
            Så den beregning kræver 3 multiplikationer og 3 additioner. 
        \end{UnderOpgave}
        Antag at et signal $x[n]$ er givet ved 
        \[x[n] = [4, 5, -6]\]
        \begin{UnderOpgave}[Beregn y1 for xn]
            \[y[1] = 4 * (-1) + 5 * 2 + (-6 * 3) = -22 + 10\]
            \[y[1] = -12\]
        \end{UnderOpgave}
        \begin{UnderOpgave}[Redegør for kausalitets og stabilitets egenskaberne af impulsresponset]
            Systemet har antikausalitet, og det medføre betyde ustabilt. 
            \[H(z) = \sum_{k = -\infty}^{\infty}h[k]z^{-k}\]
            \[H(z) = h[-1]z^{-1*(-1)} + h[0]z^0 + h[1]z^{-1}\]
            \[H(z) = 3z + 2 - z{-1}\]
            Filteret på den her form er dog en FIR, så den er stabil for alle z'er.             
        \end{UnderOpgave}
        Antag at et signal sendes gennem LTI systemet to gange. 
        \begin{UnderOpgave}[Beregn det totale impulsrespons for systemet]
            Det her tilfælde er et eksempel på en af egenskaberne for LTI systemer
            I det her tilfælde kan jeg beskrive det som et samlet system.
            \[h_2 = h[n]\star h[n] \transformation{z} H_2(z) = H(z)H(z)\]
            \[h_2 = \sum_{k = -\infty}^{\infty}{h[k]h[n - k]}\]
            \[h_2[-2]= h[-1]h[1]\]
            \[h_2[-1] = h[-1]h[-1 - (-1)] + h[0]h[-1]\]
            \[h_2[0] = h[-1]h[1] + h[0]h[0] h[1]h[-1]\]
            \[h_2[1] = 'h[-1]h[2]' + h[0]h[1] + h[1]h[0]\]
            \[h_2[1] = h[0]h[1] + h[1]h[0]\]
            \[h_2[2] = 'h[0]h[2]' + h[1]h[1]\]
            \[h_2[2] = h[1]h[1]\]

            \[h_2[n] = [h[-1]h[1], h[-1]h[0] + h[0]h[-1], h[-1]h[1] + h[0]h[0] + h[1]h[-1], h[0]h[1] + h[1]h[0], h[1]h[1]]\]
            \[h_2[n] = [3 * (-1), 3 * 2 + 2 * 3, 3 * (-1) + 2 * 2 + (-1) * 3, 2 * (-1) + (-1) * 2, (-1)*(-1)]\]
            \[===========================\]
            \[h_2[n] = [-3, 12, -2, -4, 1], \tab{0} n = -2, -1, 0, 1, 2\]
            \[===========================\]
            \[y_2[n] = \sum_{k = -\infty}^{\infty}{x[k]h_2[n - k]}\]
            Signalet er kausult. 
            \[y_2[n] = \sum_{k = 0}^{\infty}{x[k]h_2[n - k]}\]
            \[y_2[-2] = x[0]h[-2] = 4 * (-3) = -12\]
            \[y_2[-1] = x[0]h[-1] + x[1]h[-1-1] = 4 * 12 + 5 * (-3) = 48 - 15 = 33\]
            \[y_2[0] = x[0]h[0] + x[1]h[-1] + x[2]h[-2] = 4 * (-2) + 5 * (12) + (-6)*(-3) = -8 + 60 + 18 = 70\]
            \[y_3[1] = x[0]h[1] + x[1]h[0] + x[2]h[-1] = 4 * (-4) + 5 * (-2) + (-6)*(12) = - 16 - 10 - 72 = -98\]
            \[y_3[2] = x[0]h[2] + x[1]h[1] + x[2]h[0] = 4 * 1 + 5 * (-4) + (-6)*(-2) = 4 - 20 + 12 = -4\]
            \[y_4[3] = x[1]h[2] + x[2]h[1] = 5 * 1 + (-6)*(-4) = 5 - 24 = -19\]
            \[y_4[4] = x[2]h[2] = (-6)*(1) = -6\]
            \[===============================\]
            \[y[n] = [-12, 33, 70, -98, -4, -19, -6], \tab{0} n = -2, ..., 4\]
            \[===============================\]

        \end{UnderOpgave}
    \end{eksamensOpgave}
    \begin{eksamensOpgave}[LTI system design, mangler transient response svar]
        Antag at et LTI system har 
        \[H(z) = 1 + \alpha z^{-1}\]
        Der sendes et signal givet ved 
        \[x[n] = cos(\pi/4 * n), -\infty < n < \infty\]
        igennem systemet. 
        \begin{UnderOpgave}[Beregn \text{$\alpha$} således at amplituden af signalet halveres efter passage af filteret]
            Her evaluerer jeg bare filteret i den frekvens. 
            \[H(e^{j\pi/4}) = 1 + \alpha * e^{-j\pi/4}\]
            Så udvider jeg euler relationen. Jeg udnytter samtidig, at $cos(\pi/4) = sin(\pi/4) = \sqrt{2}/2$, men at sinus medføre negativt fortegn på grund af den negative frekvens.
            \[H(e^{j\pi/4}) = 1 + \alpha * \sqrt{2}/2 * (1 - j)\]
            \[|H(e^{j\pi/4})| = \sqrt{(1 + \alpha * \sqrt{2}/2)^2 + (\alpha * \sqrt{2}/2)^2} = 1/2\]
            Jeg løser den med wolfram alpha til at være. 
            \[\alpha = -\frac{1}{\sqrt{2}} \pm 0.5j\]
            Jeg har fundet dens størrelse og vinkel til approksimativt at være.
            \[\alpha = 0.8666 * e^{\pm 13/16*\pi}\]
        \end{UnderOpgave}
        \begin{UnderOpgave}[Hvordan påvirkes fasen af signalet gennem LTI systemet?]
            Hvis jeg bare skal tage udgangspunkt i en af koefficienterne.
            \[H(e^{j\omega}) = 1 + 0.8666 * e^{13/16*\pi} * e^{-j\omega}\]
            \[H(e^{j\omega}) = 1 + 0.8666 * e^{-j\omega + 13/16*\pi}\]
            Der sker intet med inputtet til n = 0, men den tidsforskudte får sin fase forskudt med $\pm 13/16\pi$               
        \end{UnderOpgave}
        Antag at signalet ikke er uendeligt langt, men derimod starter til n = 0, dvs.
        \[x[n] = cos(\pi/4 n )u[n]\]
        Hvor \text{$u[n]$} som sædvanligt betegner stepfunktionen. Den pludselige start på signalet giver anledning til en transient forvrængning. 
        \begin{UnderOpgave}[Estimer hvor mange sampels den transiente forvræning varer]
            
        \end{UnderOpgave}
        
    \end{eksamensOpgave}
    \begin{eksamensOpgave}[Lad et LTI system være givet som en parallelkobling af to 1.ordens IIR filtre]
        \begin{UnderOpgave}[Hvilke krav er der på \text{$\alpha_1, \alpha_2$} for at systemet er stabilt]
            
        \end{UnderOpgave}
        \[y[n] = y_1[n] + y_2[n]\]
        \[y_1[n] = x[n] - \alpha_2y[n - 1]\]
        \[y_2[n] = x[n] - \alpha_1y[n - 1]\]
        \[y[n] = x[n] - \alpha_2y[n - 1] + x[n] - \alpha_1y[n - 1]\]
        \[y[n] = 2x[n] - y[n - 1] * (\alpha_2 + \alpha_1)\]
        \[y[n] + y[n - 1] * (\alpha_2 + \alpha_1) = 2x[n]\]
        \[Y(z) * (1 + (\alpha_2 + \alpha_1)z^{-1}) = X(z)*2\]
        \[H(z) = Y(z)/X(z) = \frac{2}{1 + (\alpha_2 + \alpha_1)z^{-1}}\]
        Så samlet virker de som en eksponentiel funktion. Filteret er kausult, og derfor så gælder der at: 
        \[|z| > - (\alpha_2 + \alpha_1)\]
        For at systemet er stabilt. Så det er krævet for $\alpha_2, \alpha_1$ i relation til z.
        \begin{UnderOpgave}[Beregn systemfunktionen og impulsresponsen]
            Systemfunktionen har jeg allerede udledt: 
            \[H(z) = \frac{2}{1 + (\alpha_2 + \alpha_1)z^{-1}}\]
            Og jeg får at dens impulsresponse så kan beskrives som: 
            \[h[n] = 2 * (- \alpha_2 - \alpha_1)^nu[n]\]
            Bemærk jeg har brugt lineæritets princippet for et LTI system
        \end{UnderOpgave}
        \begin{UnderOpgave}[Tegn den direkte form II struktur, der svarer til LTI systemet med angivelse a og b koefficienter udtrykt ved \text{$\alpha_1, \alpha_2$}]
            Den form jeg startede med
            \[y[n] = 2x[n] - y[n - 1] * (\alpha_2 + \alpha_1)\]
            Passede på formen af et DF I systetm
            \[b = [1, 0]\]
            \[a = [1, (\alpha_2 + \alpha_1)]\]
            Og så kan jeg bare bruge dem en DF II struktur
            \fighalvfems{0.8}\\

        \end{UnderOpgave}
    \end{eksamensOpgave}
    \begin{eksamensOpgave}[Z transformation og fourier evalueret]
        Et LTI diskrete system har input x[n] og output signal y[n]. 
        \[S: y[n] = 0.5 y[n - 1] - 0.5 x[n] + x[n - 1]\]
        Lad os antage at 
        \[y[0] = 0, \tab{0} n \leq 0\]
        og lad et signal $x[n]$ være givet ved 
        \[x[n] = (-1)^n, \tab{0} n \geq 0, \tab{0} 0 ellers\]
        \begin{UnderOpgave}[Beregn for n = 1, 2, 3]
            \[y[1] = 0.5 * 0 - 0.5*(-1) + 1 = 1.5\]
            \[y[2] = 1/2 * 3/2 - (0.5 * 1) + (-1) = 3/4 - 2/4 - 4/4 = -3/4\]
            \[y[3] = -3/4 - (0.5 * (-1)) + (1) = 3/4\]
        \end{UnderOpgave}        
        \begin{UnderOpgave}[Beregn z transformationen H for overføringsfunktionen S, inkluder dens ROC]
            \[Y(z) * (1 - 0.5z^{-1}) = X(z) * (-0.5 + z^{-1})\]
            \[H(z) = Y(z)/X(z) = \frac{-0.5 + z^{-1}}{1 - 0.5z^{-1}}\]
            For at gør det nemmere at bestemme dens region of convergence, så laver jeg partial fraction på den. 
            Jeg får at den kan beskrives ved. 
            \[H(z) = \frac{-2.5}{1 - (-0.5)z^{-1}} + 2\]
            \[H(z) = -\frac{2.5}{1 + 0.5z^{-1}} + 2\]
            Systemet er kausult, så derfor har jeg en 
            \[ROC: |z| > 0.5\]
            For dc så er ROC for alle z. 
            \[====================\]
            \[H(z) = -\frac{2.5}{1 + 0.5z^{-1}} + 2, \tab{0} |z| > 0.5\]
            \[====================\]
            Dens stabilitet kan beskrives ved at se på polen. 
            \[p_0: 1 + 0.5z^{-1} = 0, \tab{0} p_0 = -0.5\]
            Alle poler indenfor enhedscirklen er stabile, og derfor vil jeg også beskrive systemet som stabilt. 
        \end{UnderOpgave}
        \begin{UnderOpgave}[Skitser magnitude responsen for \text{$H(e^{j\omega})$}]
            \figenoghalvfems{0.3}
            Magnituden er symmetrisk om $\omega = 0$ og gentager sig selv for hvert $2\pi$
        \end{UnderOpgave}
        Lad T være et LTI system
        \[T: y[n] = x[n] + \alpha x[n - 1]\]
        Hvor $\alpha \in \mathbb{R}$. 
        \begin{UnderOpgave}[Findes der et tal $\alpha$ således at kaskadekoblingen af T og S er et system med endeligt impuls svar]
            En kaskadekobling af systemer kan ses som konvolution af de to systemer og dermed gange i frekvensdomænet. 
            \[Y(z) = X(z) * (1 + \alpha z^{-1})\]
            \[H(z) = 1 + \alpha z^{-1}\]
            \[H_3(z) = \frac{-0.5 + z^{-1}}{1 - 0.5z^{-1}} * (1 + \alpha z^{-1})\]
            Så ja det gør der. Hvis jeg designer alpha så den går ud med polen fra det første system, så har jeg gjort det samlede filter til et FIR filter.\\
            =========================\\
            Så det samlede system kan blive endeligt for \\
            alpha = -0.5 \\
            =========================\\
        \end{UnderOpgave} 
    \end{eksamensOpgave}
    \begin{eksamensOpgave}[Sampling og resolution]
        Et analogt signal består af to rene sinustoner med samme amplitude. De to toner har frekvenser i $F_1 = 1200Hz$, $F_2 = 1300Hz$. Signalet $x_c(t)$ samples i 10ms.
        \begin{UnderOpgave}[Hvilken samplingfrekvens vil jeg foreslå, hvis man skal kunne adskille de to sinustoner når de ses i frekvensspektret?]
            Det her er en opgave om resolution.
            Spektret bliver dannet ud fra antal samples. Jo længere tid vi sampler, eller jo højere frekvens vi sampler med, des bedre resolution. 
            \[2\pi/N \tab{0} \leftarrow Resolution\] 
            \[N = \Delta T * F_s = 0.01s * F_s\]
            \[2\pi/(0.01s * F_s)\] 
            Og lad mig finde et udtryk for forskellen mellem de to frekvenser, men i det diskrete univers. 
            \[\omega_1 = 2\pi * 1200/f_s\]
            \[\omega_2 = 2\pi * 1300/f_s\]
            \[\Delta\omega = \omega_2 - \omega_1\]
            \[\Delta\omega = 2\pi * 1300/f_s - 2\pi * 1200/f_s\] 
            \[\Delta\omega = 2\pi * (100/f_s)\]
            \[2\pi/(0.01s * F_s) < \Delta\omega \] 
            \[2\pi/(0.01s * F_s) < 2\pi * 1300/F_s - 2\pi * 1200/F_s\] 
            \[2\pi/(0.01s * F_s) < 2\pi/F_s * (1300 - 1200)\] 
            \[2\pi/(0.01s * F_s) < 2\pi/F_s * 100\] 
            \[200\pi/F_s < 200\pi/F_s\] 
            Øhh. 
            Jeg har kludret i den. 
            Jeg har fundet formel 7.169 hvor det bliver beskrevet. 
            Han beskriver det godt nok ud fra at han bruger et vindue. 
            \[T_0 \geq \frac{1}{F_2 - F_1} = \frac{1}{1300 - 1200}\]
            \[F_s = 1/T_0 = F_2 - F_1\]
            Som minimum
            \[F_s = 100Hz\] 
            Det var godt nok ikke meget. Det kan kun være en fejl. 
            
            \[\Delta F = 100Hz\]
            \[2\pi/N = 2\pi * 100Hz\]
            \[N = 1/100\]
            \[0.01s * F_s = 1/100\]
            \[F_s = 1\]







        \end{UnderOpgave}
        
    \end{eksamensOpgave}
\end{rubrik}\setcounter{eksamensOpgave}{0}

\begin{rubrik}[Eksamenssæt2024 Ordinær - Nyt forsøg]
    \begin{eksamensOpgave}[Convolution og egenskaber]
        Strategi, forklar convolution men beregn med software. ( Hvor jeg ikke er blevet bedt om at gøre det i hånden, altså)
        \[h[n] = [3, 2, -1], \tab{0} n = -1, 0, 1\]
        \begin{UnderOpgave}[Opskriv det analytiske udtryk for at beregne y1 og angiv hvor mange multiplikationer og additioner beregninger kræver]
            Her skal jeg have gang i convolution summen. 
            \[y[n] = \sum_{k = -\infty}^{\infty}{x[k]h[n - k]}\]
            \[y[1] = \sum_{k = -\infty}^{\infty}{x[k]h[1 - k]}\]
            \[==================================\]
            \[y[1] = x[0]h[1] + x[1]h[0] + x[2]h[-1]\]
            \[\text{Så outputtet vil kræve 3 multiplikationer og 2 additioner}\]
            \[==================================\]

        \end{UnderOpgave}
        Antag at signalet xn er givet ved 
        \[x[n] = [4, 5, -6], \tab{0} n = 0, 1, 2\]
        \begin{UnderOpgave}[Beregn y1 for det xn signal]
            \[================================\]
            \[y[1] = 4 * (-1) + 5 * 2 + (-6) * 3 = -4 + 10 - 18 = -12\] 
            \[================================\]
        \end{UnderOpgave}
        \begin{UnderOpgave}[Redegor for kausalitets- og stabilitetsegenskaberne af impulsresponset hn]
            hn har antikausalitet.
            Dens stabilitet kan beskrives ud fra summen af dens impulser. 
            Den har et endeligt antal impulser og derfor er den stabil.             
        \end{UnderOpgave}
        Antag at xn sendes gennem LTI systemet to gange. 
        \begin{UnderOpgave}[Beregn det totale impulsrespons for systemet]
            Her kan man bruge system egenskaberne for LTI systemer. I z domænet kan man beskrive det samlede system ved at gange dem på hinanden. 
            I tids domæne kan man finde det samlede system som convolutionen af dem begge.
            \[h_2[n] = h[n]\star h[n] = \sum_{k = -\infty}^{\infty}{h[k]h[n - k]}\] 
            Da impuls responsen har index i n = -1, men også en i n = 1, så vil det første index være i n = -2. 
            \[h_2[-2] = h[-1]h[-2 - (-1)] = h[-1]h[-1] = 3*3 = 9\]
            \[h_2[-1] = h[-1]h[-1 - (-1)] + h[0]h[-1 + 0] = 3 * 2 + 2 * 3 = 12\]
            \[h_2[0] = h[-1]h[0 - (-1)] + h[0]h[0] + h[1]h[0 - 1] = 3 * (-1) + 2 * 2 + (-1) * 3 = 4 - 6 = -2\]
            \[h_2[1] = h[0]h[1 - 0] + h[1]h[1 - 1] = 2 * (-1) + (-1) * 2 = -4\]
            \[h_2[2] = h[1]h[2 - 1] = (-1) * (-1) = 2\]
            \[=========================\]
            \[h[n] = [9, 12, -2, -4, 2], \tab{0} n = -2, -1, 0, 1, 2\]
            \[=========================\]
        \end{UnderOpgave}
    \end{eksamensOpgave}
    \begin{eksamensOpgave}[System design]
        LTI systemet 
        \[H(z) = 1 + \alpha z^{-1}\]

        Der sendes et signal gennem: 
        \[x[n] = cos(\pi/4 * n), \tab{0} -\infty < n < \infty\]
        \begin{UnderOpgave}[Beregn alpha således at amplituden af signalet halveres efter passage af filteret]
            Ligningen der skal løses her: 
            \[solve(|H(z)| = |1 + \alpha z^{-1}| = (0.5), \alpha)\]
            Evalueret i $z = e^{j\pi/4}$
            \[|H(e^{j\pi/4})| = |1 + \alpha e^{-j\pi/4}| = (0.5)\]
            I den frekvens er $cos(\theta) = -sin(\theta) = \sqrt{2}/2$. 
            \[|H(e^{j\pi/4})| = |1 + \sqrt{2\alpha}/2 - j\sqrt{2\alpha}/2| = (0.5)\]
            \[|H(e^{j\pi/4})|^2 = (1 + \alpha\sqrt{2}/2)^2 + (\alpha\sqrt{2}/2)^2 = (0.5)^2\]
            \[c = sqrt{2}/2\]
            \[|H(e^{j\pi/4})|^2 = (1 + \alpha c)^2 + (\alpha c)^2 = (0.5)^2\]
            \[|H(e^{j\pi/4})|^2 = 1^2 + \alpha^2c^2 + 2\alpha c + \alpha^2c^2 = 0.25\]
            \[u = \alpha c\]
            \[|H(e^{j\pi/4})|^2 = 1 + 2u + 2u^2 = 0.25\]
            \[u = -0.5 \pm 0.353j\]
            \[\alpha = \frac{u}{c}\]
            Jeg løser for det, finder magnituden og vinklen og så har jeg.
            \[==========\] 
            \[\alpha \approx 0.433*e^{\pm * \frac{13}{16} \pi}\]
            \[==========\]
        \end{UnderOpgave}
        \begin{UnderOpgave}[Hvordan påvirkes fasen af signalet gennem LTI systemet]
            Inputtet til $n = 0$ er upåvirket, men signalet $n = 1$ oplever en fase ændring på $\theta_0 = \frac{13}{16}\pi$.                       
        \end{UnderOpgave}
        Antag at signalet ikke er uendelig langt, men derimod starter til $n = 0$ dvs
        \[x[n] = cos(\pi/4 * n)u[n]\]
        
        \begin{UnderOpgave}[Estimer hvor mange samples den transiente forvrænging varer]
            Den transiente forvrængning kommer på grund af, at convolutionen varmer op. Først når den mindste liste i den konvolution er fuldt beregnet i konvolutionen, så vil outputtet være i steady state.
            For det her tilfælde er x meget større end h. 
            \[h[n] = \delta[k] + \alpha \delta[k - 1] \transformation{Z} 1 + \alpha z^{-1}\]
            Så vil jeg efter $n = 0, 1$ have inputtet fuldt konvolueret med filteret. 
            Derfor regner jeg med, at det vil tage en sample at få den i steady state.
            \figtooghalvfems{0.3}\\
            Og det er det jeg ser. 
        \end{UnderOpgave}
    \end{eksamensOpgave}
    \begin{eksamensOpgave}[LTI system i parallel kobling af to to 1. ordens filtre]
        "Billede i opgavesættet"
        \begin{UnderOpgave}[Hvilke krav er der på \text{$\alpha_1, \alpha_2$} for at systemet er stabilt]
            Systemerne kan ses som 2 DF I systemer, og det vil jeg bruge senere. 
            \[y_1[n] = x[n] - \alpha_1y_1[n - 1]\]
            \[y_2[n] = x[n] - \alpha_2y_2[n - 1]\]
            \[y[n] = y_1[n] + y_2[n]\]
            Hvad jeg så umiddelbart vil sige her om stabiliteten er, at for $|\alpha_i| < 1$, så vil outputtet ikke påvirke det næste til en uendelighed. 
            Så vil gå mod en endelig værdi for inputtet begrænset til en endelig værdi. 
        \end{UnderOpgave}
        \begin{UnderOpgave}[Beregn systemfunktionen, H, og impulsresponset h]
            \[H = H_1 + H_2\]
            \[H_1 : Y_1(z)*(1 + \alpha_1z^{-1}) = X(z)\]
            \[H_1 = Y_1(z)/X(z) = \frac{1}{(1 + \alpha_1z^{-1})}\]
            \[H_2 : Y(z)*(1 + \alpha_2z^{-1}) = X(z)\]
            \[H_2 = Y_2(z)/X(z) = \frac{1}{(1 + \alpha_2z^{-1})}\]
            \[H = \frac{1}{(1 + \alpha_1z^{-1})} + \frac{1}{(1 + \alpha_2z^{-1})}\]
            Så det er to eksponentielle funktioner. Jeg antager, at systemet er kausult. 
            \[\sum_{n = -\infty}^{\infty}(a)^nu[n] = \frac{1}{1 - a}\]
            \[\sum_{n = 0}^{\infty}(a)^n = \frac{1}{1 - a}\]
            \[h[n] = (-\alpha_1)^nu[n] + (-\alpha_2)^nu[n]\]
            Kriteriet er, som jeg også før snakkede om, at størrelsen af alphaerne er mindre end 1. \\
            Mine resultater: 
            \[==================\]
            \[H = \frac{1}{(1 + \alpha_1z^{-1})} + \frac{1}{(1 + \alpha_2z^{-1})}\]
            \[h[n] = (-\alpha_1)^nu[n] + (-\alpha_2)^nu[n]\]
            \[==================\]
            
            
        \end{UnderOpgave}
        \begin{UnderOpgave}[Tegn den direkte form II der svarer til LTI systemet]
            Måske kan man beskrive det som et samlet system? 
            Ellers så vil det bare være i et parallelt system som det her. 
            Jeg brug sympy til at udvide udtrykket så jeg får.
            \[H(z) = \frac{2 + (\alpha_1 + \alpha_2)* z^{-1}}{1 + (\alpha_1 + \alpha_2)z^{-1} + \alpha_1\alpha_2 z^{-2}}\]
            Jeg kan beskrive systemet så som: 
            \[b = [2, (\alpha_1 + \alpha_2), 0]\]
            \[a = [1, (\alpha_1 + \alpha_2), \alpha_1\alpha_2]\]
            Med den opsætning, så laver jeg systemet. 
            \figtreoghalvfems{0.4}\vspace{100pt}       
        \end{UnderOpgave}
    \end{eksamensOpgave}
    \begin{eksamensOpgave}[LTI system karakteristikker]
        \[S: y[n] = 0.5y[n - 1] - 0.5x[n] + x[n - 1]\]
        \[x[n] = (-1)^n, \tab{0} n \geq 0\]
        \begin{UnderOpgave}[Beregn y1, y2 og y3]
            \[y[1] = 0 - 0.5(-1)^1 + (-1)^0 = 3/2\]
            \[y[2] = 1/2 * 3/2 - 0.5 * (-1)^2 + (-1)^1 = 3/4 - 1/2 - 1 = -3/4\]        
            \[y[3] = 1/2*(-3/4) - 0.5 * (-1)^3 + (-1)^2 = -3/8 + 1/2 + 1 = 9/8\]
        \end{UnderOpgave}
        \begin{UnderOpgave}[Beregn z transformationen H for overføringsfunktionen S, inklusiv dens ROC]
            \[Y(z)*(1 - 0.5z^{-1}) = X(z)*(-0.5 + z^{-1})\]
            \[H(z) = \frac{-0.5 + z^{-1}}{1 - 0.5z^{-1}}\]
            Jeg laver partial fraction for at bedre kunne beskrive dens transformationer.
            \[H(z) = 1.5 * \frac{1}{1 - 0.5z^{-1}} - 2\]
            Så har jeg en DC værdi, som kommer fra en impuls. Det er FIR og har ROC i alle z. 
            En eksponentiel funktion. Systemet er kausult, så får den gælder der at 
            \[ROC: |z| > 0.5\]
            \[H(z) = \frac{-0.5 + z^{-1}}{1 - 0.5z^{-1}} = 1.5 * \frac{1}{1 - 0.5z^{-1}} - 2, \tab{0} ROC: |z| > 0.5\]
            Systemets feedback har en værdi under 1, så den vil aftage med tiden. Med det in mente, så er den stabil så længe inputtet er stabilt. 

        \end{UnderOpgave}
        \begin{UnderOpgave}[Skitser magnituden evalueret i $z=e^{j\omega}$]
            \figfireoghalvfems{0.4}
            Jeg plottede også lige for fasen. Filterets magnitude er konstant men fasen ændres.
        \end{UnderOpgave}
        Et nyt system 
        \[T\colon y[n]=x[n]+\alpha\cdot x[n - 1],\]
        \begin{UnderOpgave}[Findes der en alpha der gør at T og S i en kaskade kobling resulterer i et system med en endelig impuls]
            Endelig mener her IIR -> FIR filter. 
            Det kan sagtens være. Kaskadekobling i frekvens spektret medføre multiplikation. 
            \[H_2 : Y(z) = X(z)*(1 + \alpha z^{-1})\]
            \[H_2 = 1 + \alpha z^{-1}\]
            \[H_3 = H * H_2 = \frac{-0.5 + z^{-1}}{1 - 0.5z^{-1}} * (1 + \alpha z^{-1})\]
            Så ja det kan den sagtens. For at det skal ske, så skal nævneren udlignes. Altså det nye nulpunkt skal sættes så polen udlignes. 
            Og det gør den for
            \[\alpha = -0.5\]
            Hvor 
            \[H_3(z) = -0.5 + z^{-1}, \tab{0} \alpha = -0.5\]
        \end{UnderOpgave}
    \end{eksamensOpgave}

    \begin{eksamensOpgave}[Sampling, vindue og DFT]
        Et analogt signal xc(t) er en superposition af 2 rene sinustoner, med samme amplitude. De to toner har frekvenser på helholdsvis 
        \[F_1 = 1200Hz, F_2 = 1300Hz\]
        Der samples i $T_0 = 10ms$
        \begin{UnderOpgave}[Hvilken samplingfrekvens vil jeg anbefale for at man kan adskille de to sinustoner i frekvensspektret]
            \[\Delta F = F_2 - F_1 = 100Hz\]
            Spektret er over $F: [-1, 1]$ normaliseret. 
            Og den er opdelt i N samples 
            \[N = T_0 * F_s\]
            Så resolutionen er 
            \[(1 - (-1))/(T_0 * F_s)\]
            Hvad er afstanden i det normaliserede spekter? 
            \[\Delta_F/F_S\] 
            Og så burde det, og jeg lægger vægt på burde, kunne skrives sådan her.
            \[(2)/(T_0 * F_s) = \Delta_F/F_S\]
            Men jeg ser, at det er to udtryk fra samme ligning, så de går ud. 
            
        \end{UnderOpgave}
        \begin{UnderOpgave}[Hvilken indflydelse vil et vinduesfunktion have på mit svar i spg. 1. Hvilket vindue kan jeg anbefale?]
            For en vindue så vil den bredde spektret. Det gør den på grund af den modelerede måde de bliver lavet på. 
            For Hann husker jeg den som en impuls i 0 og ellers $\pm$ frekvens fra cos funktion. 
            For hver sinustone vil jeg så have modulation ud fra den nye moduleringsfrekvens. Så var det måske smart at visualisere det, for så at svare på det. 
        \end{UnderOpgave}
    \end{eksamensOpgave}
\end{rubrik}

